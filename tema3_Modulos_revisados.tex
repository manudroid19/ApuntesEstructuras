

%-----------------------------------------------------------------------------------------------------
%	INCLUSIÓN DE PAQUETES BÁSICOS.
%-----------------------------------------------------------------------------------------------------

\documentclass{article}
\usepackage{enumerate}
\usepackage{wrapfig}
\usepackage{graphicx}



%-----------------------------------------------------------------------------------------------------
%	SELECCIÓN DEL LENGUAJE
%-----------------------------------------------------------------------------------------------------

% Paquetes para adaptar Látex al Español:
\usepackage[spanish,es-noquoting, es-tabla, es-lcroman]{babel} % Cambia
\usepackage[utf8]{inputenc}                                    % Permite los acentos.
\selectlanguage{spanish}                                       % Selecciono como lenguaje el Español.

%flechassssss
\def\flechaSobreyectiva{\mathrel{\mkern16mu  \vcenter{\hbox{$\scriptscriptstyle+$}}%
		\mkern-25mu{\longrightarrow}}}
\def\flechaInyectiva{\mathrel{\mkern0mu  \vcenter{\hbox{$\scriptscriptstyle+$}}%
		\mkern-9mu{\longrightarrow}}}
\def\flechaBiyectiva{\mathrel{\mkern16mu  \vcenter{\hbox{$\scriptscriptstyle+$}}\mkern-25mu{\mathrel{\mkern0mu\vcenter{\hbox{$\scriptscriptstyle+$}}\mkern-9mu{\longrightarrow}}}}}
\def\xFlechaInyectiva #1{\mathrel{\ooalign{\thinspace\thinspace$\mapstochar\mkern5mu$\hfil\cr$\xrightarrow{#1}$\cr}}}
\def\xFlechaSobreyectiva #1{\mathrel{\ooalign{\hfil$\mapstochar\mkern5mu$\thinspace\thinspace\cr$\xrightarrow{#1}$\cr}}}
%llave a la derecha
\newenvironment{rcases}
{\left.\begin{aligned}}
	{\end{aligned}\right\rbrace}

%-----------------------------------------------------------------------------------------------------
%	SELECCIÓN DE LA FUENTE
%-----------------------------------------------------------------------------------------------------

% Fuente utilizada.
\usepackage{courier}                    % Fuente Courier.
\usepackage{microtype}                  % Mejora la letra final de cara al lector.

%-----------------------------------------------------------------------------------------------------
%	ESTILO DE PÁGINA
%-----------------------------------------------------------------------------------------------------

% Paquetes para el diseño de página:
\usepackage{fancyhdr}               % Utilizado para hacer títulos propios.
\usepackage{lastpage}               % Referencia a la última página. Utilizado para el pie de página.
\usepackage{extramarks}             % Marcas extras. Utilizado en pie de página y cabecera.
\usepackage[parfill]{parskip}       % Crea una nueva línea entre párrafos.
\usepackage{geometry}               % Asigna la "geometría" de las páginas.

% Se elige el estilo fancy y márgenes de 3 centímetros.
\pagestyle{fancy}
\geometry{left=3cm,right=3cm,top=3cm,bottom=3cm,headheight=1cm,headsep=0.5cm} % Márgenes y cabecera.
% Se limpia la cabecera y el pie de página para poder rehacerlos luego.
\fancyhf{}

% Espacios en el documento:
\linespread{1.1}                        % Espacio entre líneas.
\setlength\parindent{0pt}               % Selecciona la indentación para cada inicio de párrafo.

% Cabecera del documento. Se ajusta la línea de la cabecera.
\renewcommand\headrule{
	\begin{minipage}{1\textwidth}
		\hrule width \hsize
	\end{minipage}
}

% Texto de la cabecera:
\lhead{\docauthor}                          % Parte izquierda.
\chead{}                                    % Centro.
\rhead{\subject \ - \doctitle}              % Parte derecha.

% Pie de página del documento. Se ajusta la línea del pie de página.
\renewcommand\footrule{
	\begin{minipage}{1\textwidth}
		\hrule width \hsize
	\end{minipage}\par
}

\lfoot{}                                                 % Parte izquierda.
\cfoot{}                                                 % Centro.
\rfoot{Página\ \thepage\ de\ \protect\pageref{LastPage}} % Parte derecha.

%----------------------------------------------------------------------------------------
%	MATEMÁTICAS
%----------------------------------------------------------------------------------------

% Paquetes para matemáticas:
\usepackage{amsmath, amsthm, amssymb, amsfonts, amscd} % Teoremas, fuentes y símbolos.

\usepackage[all]{xy} %para diagramas conmutativos

% Nuevo estilo para definiciones
\newtheoremstyle{definition-style} % Nombre del estilo
{5pt}                % Espacio por encima
{5pt}                % Espacio por debajo
{}                   % Fuente del cuerpo
{}                   % Identación: vacío= sin identación, \parindent = identación del parráfo
{\bf}                % Fuente para la cabecera
{.}                  % Puntuación tras la cabecera
{.5em}               % Espacio tras la cabecera: { } = espacio usal entre palabras, \newline = nueva línea
{}                   % Especificación de la cabecera (si se deja vaía implica 'normal')

% Nuevo estilo para teoremas
\newtheoremstyle{theorem-style} % Nombre del estilo
{5pt}                % Espacio por encima
{5pt}                % Espacio por debajo
{\itshape}           % Fuente del cuerpo
{}                   % Identación: vacío= sin identación, \parindent = identación del parráfo
{\bf}                % Fuente para la cabecera
{.}                  % Puntuación tras la cabecera
{.5em}               % Espacio tras la cabecera: { } = espacio usal entre palabras, \newline = nueva línea
{}                   % Especificación de la cabecera (si se deja vaía implica 'normal')

% Nuevo estilo para ejemplos y ejercicios
\newtheoremstyle{example-style} % Nombre del estilo
{5pt}                % Espacio por encima
{5pt}                % Espacio por debajo
{}                   % Fuente del cuerpo
{}                   % Identación: vacío= sin identación, \parindent = identación del parráfo
{\scshape}                % Fuente para la cabecera
{:}                  % Puntuación tras la cabecera
{.5em}               % Espacio tras la cabecera: { } = espacio usal entre palabras, \newline = nueva línea
{}                   % Especificación de la cabecera (si se deja vaía implica 'normal')

% Teoremas:
\theoremstyle{theorem-style}  % Otras posibilidades: plain (por defecto), definition, remark
\newtheorem{theorem}{Teorema}[section]  % [section] indica que el contador se reinicia cada sección
\newtheorem{corollary}[theorem]{Corolario} % [theorem] indica que comparte el contador con theorem
\newtheorem{lemma}[theorem]{Lema}
\newtheorem{proposition}[theorem]{Proposición}

% Definiciones, notas, conjeturas
\theoremstyle{definition}
\newtheorem{definition}{Definición}[section]
\newtheorem{conjecture}{Conjetura}[section]
\newtheorem*{note}{Nota} % * indica que no tiene contador
\newtheorem*{observation}{Observación} % * indica que no tiene contador
\newtheorem*{properties}{Propiedades}
\newtheorem*{comment}{Comentario clase}

% Ejemplos, ejercicios
\theoremstyle{example-style}
\newtheorem{example}{Ejemplo}[section]
\newtheorem{exercise}{Ejercicio}[section]

%-----------------------------------------------------------------------------------------------------
%	PORTADA
%-----------------------------------------------------------------------------------------------------

% Elija uno de los siguientes formatos.
% No olvide incluir los archivos .sty asociados en el directorio del documento.
%\usepackage{title1}
\usepackage{title2}
%\usepackage{title3}

%-----------------------------------------------------------------------------------------------------
%	TÍTULO, AUTOR Y OTROS DATOS DEL DOCUMENTO
%-----------------------------------------------------------------------------------------------------

% Título del documento.
\newcommand{\doctitle}{Tema 3}%git clone https://github.com/manudroid19/ApuntesEstructuras.git
% Subtítulo.
\newcommand{\docsubtitle}{Módulos}
% Fecha.
\newcommand{\docdate}{}
% Asignatura.
\newcommand{\subject}{Estructuras Algebraicas}
% Autor.
\newcommand{\docauthor}{Manuel de Prada, Jorge Vázquez, Sergio Mayo}
\newcommand{\docaddress}{}
\newcommand{\docemail}{}

%-----------------------------------------------------------------------------------------------------
%	RESUMEN
%-------------------------------					----------------------------------------------------------------------

% Resumen del documento. Va en la portada.
% Puedes también dejarlo vacío, en cuyo caso no aparece en la portada.
%\newcommand{\docabstract}{}
\newcommand{\docabstract}{}

\begin{document}


%-----------------------------------------------------------------------------------------------------
%	ÍNDICE
%-----------------------------------------------------------------------------------------------------

% Profundidad del Índice:
%\setcounter{tocdepth}{1}

\newpage
\tableofcontents
\newpage

%----------------------------------------------------------------------------------------
%	Sección 1: Deficiones y teoremas
%----------------------------------------------------------------------------------------

\section{Módulos}
	Sea $A$ un anillo conmutativo.
	\begin{definition}
        	Un $A$-módulo es un grupo abeliano $M$ con una operación:
        	\[A\times M\longrightarrow M \]
        	\[(a,m) \longmapsto a\cdot m =am\]
        	que verifica:
        	\begin{enumerate}[\hspace{1cm}i)]
        		\item $(a+b)m = am+bm$
        		\item $a(m+n)=am+an$
        		\item $(ab)m = a(bm)$
        		\item $1m=m$
        	\end{enumerate}
	\end{definition}
	\ 

	\begin{observation}
	    	$A$ es un $A$-módulo
	    	\[A\times A\longrightarrow A \]
	    	\[(a,b) \longmapsto a\cdot b\]
	\end{observation}
	\ 

	\begin{properties}
		\
	    	\begin{enumerate}
	    		\item $0m=0$ ya que $0m=(0+0)m=0m+0m$
	    		\item $a0=0$
	    		\item $(-a)m=-(am)=a(-m)$
		\end{enumerate}
	\end{properties}
	\ 

	\begin{observation}
		$am=0 \nRightarrow \begin{cases}
		a=0\\
		\quad\text{ó}\\
		m=0
		\end{cases}$
	\end{observation}
	\ 

	\begin{observation}
		Como $A$ conmutativo, $ma:=am$
		\[m(ab)=(ab)m=a(bm)=mb(a)=m(ba)\]
	\end{observation}
	\begin{proposition}
		Sea $M$ grupo abeliano.
		Entonces $M$ es un $\mathbb{Z}$-módulo (la estructura de $\mathbb{Z}$-módulo y de grupo abeliano es la misma).
	\end{proposition}
	\begin{proof}
		Sea $n \in \mathbb{Z}, x\in M$
		\[n\cdot x= \begin{cases}
			x+...+x&\quad \text{si }n>0\\
			0 &\quad \text{si } n=0\\
			(-x)+...+(-x)=-(-n)x &\quad \text{si }n<0
		\end{cases}
		 \]
		$nx=(1+...+1)x=1x+...1x=x+...+x$
	\end{proof}


	\begin{definition}
		Sea $M$ un $A$-módulo, $N\subset M$.
		Se dice que $N$ es un $A$-submódulo de $M$ si $N$ es un $A$-módulo con la operación de $A$ en $M$.
	\end{definition}

	\begin{properties}
		\ 
		\begin{enumerate}
			\item $ y \in N, a \in A \Rightarrow  ay\in N$
			\item $ x,y\in N  \Rightarrow x-y \in N$
			\item $\{0\}, M$ son submódulos de $M$
			\item Sean $I$ un ideal de $A$ y $M$ un $A$-módulo. Entonces:
			\[IM:=\{ \sum_{i=1}^{n} a_ix_i \quad : n \in N, a_i \in I, x_i \in M\}\]
			es un submódulo de M. ($a  \sum a_i x_i = \sum (a a_i)  x_i$, nótese que $aa_i\in I$)

			\item Considerando $A$ como $A$-módulo se tiene que, dado $I\subset A$, $I$ es $A$-submódulo $\Leftrightarrow I$ es ideal de A.
		\end{enumerate}
	\end{properties}
	\

	\begin{definition}
		Si $ M_1 $ y $ M_2 $ son submódulos de un $ A $-módulo M se define la suma de $M_1 $ y $ M_2$:
				\[ M_1+M_2 = \{x_1+x_2 \in M : x_1\in M_1, x_2 \in M_2\}\]
		Es el menor submódulo de $ M $ que contiene a $ M_1 $ y $ M_2 $.\\
		Nótese:
		\begin{enumerate}
			\item $ x_1+x_2-(x_1'+x_2') = (x_1-x_1')+ (x_2-x_2') \in M_1+M_2 $
			\item $ a(x_1+x_2)=a(x_1)+a(x_2)\in M_1+M_2 $
		\end{enumerate}
	\end{definition}
	\ 

	\begin{definition}
			Sea $ \{M_i\}_{i\in I} $ familia de submódulos de $ M $. Definimos en general
			\begin{align*}
				\sum_{i\in I}M_i &= \{\sum_{i\in I}x_i \in M : x_i\in M_i \text{para cada i, }a_i\in A, \text{casi todos los }x_i=0\}\\
				&= \{\text{sumas finitas de elementos de }\bigcup M_i\}=x_1+...+x_r
			\end{align*}

			Es el menor submódulo de $ M $ que contiene $ \bigcup_{i\in I} M_i $

	\end{definition}
	\ 

	\begin{proposition}
		La intersección de una familia de submódulos de M es un submódulo de M.
	\end{proposition}

	\begin{proposition}
		Sea $M$ un $A$-módulo y $X \subset M$. Se llama $A$-submódulo de $M$ generado por $X$ al menor $A$-submódulo de $M$ que contiene a $X$.
		Se denota $\langle X\rangle$ o $\langle X\rangle_{A}$. $\\$
		Se dice que $X$ es un conjunto de generadores de $\langle X\rangle$.
		\[\langle X\rangle = \{\sum_{i = 1}^{n}a_i x_i\  /\  n\in\mathbb{N},a_i \in A, x_i\in X\}\]
	\end{proposition}

	\begin{example}
		\ 

		\begin{enumerate}
			\item Si $X = {x}$ tenemos que:
			\[\langle \{x\}\rangle = \langle x\rangle = Ax =\{ax/a\in A\}\]
			\item 	$\langle X\rangle = \sum_{x\in X}Ax$
			\item 	$\sum_{i}  M_i = \langle \bigcup_i M_i\rangle$
		\end{enumerate}
	\end{example}
	\ 

	\begin{definition}
		Sean $M$, $N$ $A$-módulo. Una aplicación $f:M \rightarrow N$ es un homomorfismo de $A$-módulo si:
		\begin{enumerate}
		\item  $f(x+y) = f(x)+f(y)$
		\item $f(ax) = af(x)$
		\end{enumerate}
	\end{definition}
	\ 

	\begin{proposition}
		La composición de homomorfismos de $A$-módulos es un homomorfismo de $A$-módulos.
	\end{proposition}
	\ 

	\begin{example} Tenemos que:
		\begin{enumerate}
			\item $id_M : M \rightarrow M$ es un homomorfismo de módulos.
			\item $f: M \rightarrow N$ $f(x) = 0$ $\forall x \in M$ es un homomorfismo de $A$-módulo.
		\end{enumerate}
	\end{example}
	\ 

	\begin{observation}
		Es necesario destacar que de forma inmediata que se cumple que:\\ $f(0) = 0$, $f(-x) = -f(x)$.
	\end{observation}
	\ 

	\begin{proposition}
		Sea $f: M \rightarrow N$ un homomorfismo de $A$-módulo. Se cumple:
		\begin{enumerate}
			\item $M'  \subset M$ un submódulo $\Rightarrow f(M')$ es un submódulo de $N$.
			\item $N' \subset N$ un submódulo $\Rightarrow f^{-1}(N')$ es un submódulo de $M$.
			\item $ker(f) = f^{-1}(0)$ es un submódulo de $M$.
			\item $Im(f) = f(M)$ submódulo de $N$.
			\item $f$ inyectiva  $\Leftrightarrow ker(f) = 0$.
		\end{enumerate}
	\end{proposition}
	\ 

	\begin{proposition}
		Sea $M$ y $N$ un $A$-submódulo de $M$. En $M/N$ se cumple:
		\begin{enumerate}
			\item$(x + N) + (y + N) =(x +y) + N$
			\item $a(x + N) = ax + N $
			\item Si $x + N = x\prime +N$ tenemos que  $ax +N=ax\prime +N$
		\end{enumerate}
	\end{proposition}

	\begin{proof}
		\[x=x\prime \Leftrightarrow x-x\prime \in N \Rightarrow a(x +x\prime)\in N \Rightarrow ax+ax\prime \in N \Leftrightarrow ax+N=ax\prime+N\]
	\end{proof}

	\begin{observation}
		Tenemos que $\pi : M \rightarrow M/N$ homomorfismo de $A$-módulos y $ker(f) = N$ ($M=N \Rightarrow M/N=\{0\}$).
	\end{observation}


	\begin{proposition}
	Sea $N$ un submódulo de $M$. Existe una biyección que conserva las inclusiones de la forma:


		\begin{center}
			$\{A$-submódulos de $M$ que contienen a $N\} \rightleftarrows \{A$-submódulo de $M/N\}]$.
		\end{center}

	\end{proposition}

	\begin{definition}
		Si $f: M\rightarrow N$ es un homorfismo de $A$-módulos se llama conúcleo de $f$ al
		$A$-módulo:
		\[coker(f) = M/Im(f).\]
	\end{definition}

	\begin{definition}
		Un homomorfismo $f:M\rightarrow N$ de $A$-módulos es un isomorfismo si existe un homorfismo de $A$-módulos $g:N\rightarrow M / g\circ f = id_M$, $f\circ g = id_N$. En este caso es $g$ único y $g=f^{-1}$.$\\$
		Así, si existe un isomorfismo de $M$ a $N$ se dice que $M$ y $N$ son isomorfos y se denota como $M\simeq N$.
	\end{definition}

	\begin{exercise}
		$f$ isomorfismo $\Leftrightarrow$ $f$ homomorfismo biyectivo.
	\end{exercise}

	Ahora veremos un resultado que ya habíamos observado en grupos, por ejemplo, y que es el representado en el siguiente esquema:
	\[
	\xymatrix{
		M \ar[d]^{\pi} \ar[r]^f & N \\
		M/ker(f) \ar[r]^{f'} & Im(f) \ar@{^{(}->}[u]^i
	}
	\]

	Ya vimos como $\pi$ es un homomorfismo de módulos. Ahora, queda ver que $f'$ es un isomorfismo, es decir, que $M/ker(f)\simeq Im(f)$.
	\begin{exercise}
		$f'$ es un isomorfismo.
	\end{exercise}
	\begin{proposition}
		Sean $M_1\subset M_2\subset M$ $A$-submódulos. Entonces:
		\[\frac{M/M_1}{M_2/M_1}\simeq M/M_2\]
	\end{proposition}


	\begin{proof}
		Tomemos la aplicación $f:M/M_1 \rightarrow M/M_2$ que se define como:
		\[f(x + M_1) = x + M_2\]
		y veamos que es un homomorfismo.
		\begin{enumerate}
			\item $x + M_1 = y + M_1 \Rightarrow x-y\in M_1  \stackrel{M_1\subset M_2}{\Rightarrow}  x-y \in M_2 \Rightarrow x + M_2 = y + M_2$
			\item $f((x+M_1)+(y + M_1)) = f(x+y+M_1) = x+y+M_2 = (x+M_2)+(y+M_2) = f(x+M_1) + f(y+M_1)$
			\item $f(a(x+M_1)) = f(ax + M_1) = ax + M_2 = a(x + M_2) = af(x+M_1)$
		\end{enumerate}
		Una vez visto que, efectivamente, $f$ es un homomorfismo, vemos que es sobreyectivo, es decir, que la imagen es exactamente $M/M_2$, de forma inmediata. Para finalizar, veamos que $ker(f) = M_2/M_1$. Si $x+M_1\in ker(f)$, tenemos que:
		\[f(x + M_1) = 0 + M_2 \Leftrightarrow f(x + M_1) = x + M_2,\  x \in M_2\]
		\[x+M_1 \in M_2/M_1 \Leftrightarrow x \in M_2\]
		de donde se extrae de forma inmediata lo que queríamos prooverline.
		Por lo tanto, aplicando lo observado en el anterior diagrama conmutativo, hemos probado que:
		\[\frac{M/M_1}{ker(f)} = \frac{M/M_1}{M_2/M_1} \simeq M/M_2.\]
	\end{proof}

	\begin{proposition}
		Sea $M$ un $A$-módulo y $M_1$, $M_2$ $A$-submódulos. Entonces:
		\[\frac{M_1+M_2}{M_1}\simeq \frac{M_2}{M_1\cap M_2}\]
	\end{proposition}
	\begin{proof}
		Para prooverline esto tomamos:
		\[
		\xymatrix{
			M_2 \ar[rr]^f \ar@{^{(}->}[rd] & & \frac{M_1+M_2}{M_1} \\
			& M_1 + M_2\ar[ru] \\
		}
		\]
		siendo $f$ un homomorfismo de $A$-módulos con $f(x) = x + M_1$. Esta aplicación es sobreyectiva ya que $y+x+M_1 \in \frac{M_1+M_2}{M_1}$ con $y \in M_1$ y $x \in M_2$, pero entonces $y + x + M_1 = x + M_1$. $\\$
		Además, su núcleo es $M_1 \cap M_2$ (exactamente igual que la demostración para grupos). Por lo tanto:
		\[\frac{M_1+M_2}{M_1}\simeq \frac{M_2}{M_1\cap M_2}\]
	\end{proof}

	\begin{definition}
		Sea $A$ un anillo. Un $A$-módulo $M$ es cíclico si está generado por un elemento, es decir:
		\[\exists x  \in M \ / \   M=\langle x\rangle=Ax\]
	\end{definition}

	\begin{observation}
		Sea $f:A\rightarrow B$ un homomorfismo de anillos. Entonces $B$ es un $A$-módulo definiendo la operación
		externa como:\[
			a\in A,b\in B \ \ \ \ a\cdot b :=f(a)b
		\]
		En particular si $I$ es un ideal de $A$, $A/I$ es un $A$-módulo ($p:A\rightarrow A/I$).
	\end{observation}

	\begin{proposition}
		Si $M$ es un $A$-módulo son equivalentes:
		\begin{enumerate}[\hspace{1cm}i)]
			\item $M$ es cíclico.
			\item Existe un homomorfismo sobreyectivo de $A$-módulos $\phi :A\rightarrow M$
			\item Existe un ideal $I$ de $A$ y un isomorfismo de $A$-módulos $A/I \simeq M$
		\end{enumerate}
	\end{proposition}

	\begin{proof} \ \\
		$i)\Rightarrow ii)$:
		\[ M \text{ cíclico}\Rightarrow M=\langle x\rangle\]
		\[ \phi :a \in A\rightarrow \phi (a):=ax \in M \text{ es un homomorfismo de $A$-módulos sobreyectivo:}\]
		\[ \phi(a+b)=(a+b)x=ax+bx=\phi(a)+\phi(b)\]
		\[ \phi(\lambda a)=(\lambda a)x= \lambda(ax)=\lambda\phi(a) \ \ \ \forall\lambda\in A \]
		\[ Im(\phi)=\{ax \ / \ a\in A \} = \langle x\rangle=M\]
		$ii)\Rightarrow iii)$:
		\[ \text{Por hipótesis existe } \phi:A\rightarrow M \text{ homomorfismo sobreyectivo de $A$-módulos.}\]
		\[ A/Ker(\phi) \simeq M \text{ y $Ker(\phi)$ es un ideal de } A \]
		$iii)\Rightarrow i)$:
		\[ A/I \text{ es un $A$-módulo cíclico.} \]
		\[ A/I=\langle \overline{1}\rangle\ \Rightarrow\text{$M$ es cíclico} \]
	\end{proof}

	\begin{definition}
		Sean $A$ anillo y $M$ $A$-módulo. Se llama aniquilador de $M$ al ideal de $A$ definido por:
		\[(0:M):=\{a\in A \ / \ ax=0 \  \forall x \in M\}\]
	\end{definition}

	\begin{definition}
		Tomando $M=\langle x\rangle$ definimos:
		\[(0:x):=(0:\langle X\rangle)=\{a\in A \ / \ ax=0\}\]
	\end{definition}

	\begin{proposition}
		Sea $A$ un anillo y $M$ un $A$-módulo. Se verifica que en efecto $(0:M)$ es un ideal.
	\end{proposition}

	\begin{proof}
		\[ \text{Trivialmente } 0 \in (0:M) \Rightarrow (0:M) \neq \text{\O}\]
		\[ a,b \in (0:M) \Rightarrow ax=0=bx \ \forall x \in M \Rightarrow (a-b)x=ax-bx=0-0=0 \ \forall x \in M \Rightarrow a-b \in (0:M) \]
		\[ \lambda \in A, a \in (0:M) \Rightarrow ax=0 \ \forall x \in M \Rightarrow (\lambda a)x=0 \ \forall x \in M \Rightarrow \lambda a \in (0:M) \]
	\end{proof}

	\begin{observation}
		Sea $x \in M$, definimos:
		\[ A\xrightarrow{\ \ \phi \ \ } M \text{ homomorfismo de $A$-módulos}\]
		\[ \phi(a):=ax\]
		\[ Ker(\phi)=(0:x)\]
		\[A/(0:x)\xrightarrow{\ \ \cdot x\ \ }M \text{ homomorfismo inyectivo}\]
	\end{observation}

	\begin{observation}
		Tomando $A=\mathbb{Z}$:
		\[ \mathbb{Z}\text{-módulo}=\text{ grupo abeliano}\]
		\[ \mathbb{Z}\text{-módulo cíclico}=\text{ grupo cíclico}\]
	\end{observation}

	\begin{definition}
		Sea $\{M_i\}_{i\in I}$ una familia de $A$-módulos. Definimos el producto directo de los $M_i$ como el $A$-módulo:
		\[ \prod_{i\in I}M_i := \{f:I\rightarrow\bigcup_{i\in I}M_i \ / \ \text{$f$ aplicación}, f(i)\in M_i \ \forall i \in I\} = \{(x_i)_{i\in I} \ / \ x_i \in M_i \ \forall i \in I\}\]
		\[(x_i)_{i\in I} + (y_i)_{i\in I} := (x_i + y_i)_{i\in I}\]
		\[a(x_i)_{i\in I} := (ax_i)_{i\in I}\]
	\end{definition}

	\begin{observation}
		En el caso finito, los elementos del producto los representamos con n-tuplas:
		\[(x_1,\dots,x_n)+(y_1,\dots,y_n)=(x_1+y_1,\dots,x_n+y_n)\]
		\[a(x_1,\dots,x_n)=(ax_1,\dots,ax_n)\]
	\end{observation}

	\begin{observation}
		La proyección habitual $\pi_j:\prod_{i\in I}M_i \longrightarrow M_j$ es un homomorfismo de $A$-módulos para todo $j$.
	\end{observation}

	\begin{definition}
		Sea $\{M_i\}_{i\in I}$ una familia de $A$-módulos. Definimos la suma directa de los $M_i$ como el $A$-submódulo de $\prod_{i\in I}M_i$:
		\[\bigoplus_{i\in I}M_i=\{(x_i)_{i\in I} \in \prod_{i\in I}M_i \ / \ x_i=0 \text{ para casi todo } i \in I\}\]
	\end{definition}

	\begin{observation}
		La aplicación
		\[M_j:M_j\longrightarrow\bigoplus_{i\in I}M_i\]
		\[M_j(m):= (x_i)_{i\in I} \ / \ x_j=m,\  x_i=0 \ \forall i\neq j \]
		es un homomorfismo de $A$-módulos inyectivo para cualquier $j$.
	\end{observation}

	\begin{observation}
		Si $I$ es finito:
		\[\prod_{i\in I}M_i = \bigoplus_{i\in I}M_i \]
	\end{observation}

	\begin{proposition}
		Sea $M$ un $A$-módulo y $\{M_i\}_{i\in I}$ una familia de $A$-submódulos de $M$. La aplicación:
		\[ \bigoplus_{i\in I}M_i \longrightarrow \sum_{i\in I}M_i \]
		\[ (x_i)_{i\in I} \leadsto \sum_{i\in I}x_i \]
		es un homomorfismo de $A$-módulos.
	\end{proposition}

	\begin{proposition}
		Sea $ \{M_i\}_{i\in I} $ familia de submódulos de $ M $. Consideremos:
		\[\phi : \bigoplus_{i\in I} M_i\longrightarrow \sum_{i\in I }M_i \]
		\[\phi ((x_i)_{i\in I})=\sum_{i\in I} x_i \]
		Equivalen:
		\begin{enumerate}[\hspace{1cm}i)]
			\item $\phi$ isomorfismo
			\item $\sum_{i\in I}x_i=0 \ \  (x_i \in M_i, \ x_i=0$ para casi todo i$)$ $\Rightarrow$ $ x_i=0 \ \forall i \in I $
			\item $M_j \cap (\sum_{i\neq j}M_i) = 0 \ \forall j \in I$
		\end{enumerate}
	\end{proposition}

	\begin{proof}
		\ 

		$i) \Rightarrow ii)$:

		 \[ 0=\sum_{i\in I} x_i=\phi ((x_i)_{i\in I}) \Rightarrow \text{(ya que $\phi$ inyectiva) } (x_i) \in I =0 \Rightarrow x_i =0 \ \forall i \in I \]
		$ii) \Rightarrow iii)$:

		\[ x \in M_j \cap (\sum_{i\neq j}M_i) \Rightarrow x=\sum_{\substack{i\in I, i\neq j \\ x_i\in M_i}} x_i \Rightarrow\]
		 \[ \sum_{i\in I, i\neq j} x_i -x = 0 \  (\text{ya que }x\in M_j) \Rightarrow  x=0  \]

		$iii) \Rightarrow i)$:

		\parshape 1 2cm \dimexpr\linewidth-2cm\relax
		Sea $ (x_i)_{i\in I} \in \oplus_{i\in I} M_i  $ tal que $ \phi ((x_i)_{i\in I})=\sum_{i\in I} x_i=0 $. \\
		Tenemos que demostrar que $ x_i=0 \ \forall i \in I $. \\
		Sea $ j\in I, x_j = -\sum_{i\in I, i\neq j}x_i \in \sum_{i\in I} M_j \Rightarrow x_j \in M_j \cap (\sum_{i\neq j} M_i)=0 \Rightarrow x_j=0 $
	\end{proof}

	\begin{definition}
		$ A $ dominio. Sea $ M$ un $ A $-módulo. Un elemento $ x\in M $ es de torsión si $ (0:x)\neq 0 $.
		(si existe $ a\in A, a \neq 0 $ tal que $ ax=0 $).
		El conjunto de elementos de torsión de $ M $ se denota $ T(M) $ y es un $ A $-submódulo de $ M $.
	\end{definition}

	\begin{observation}
		\[x,y\in T(M) \Rightarrow \exists a,b \in A, a\neq 0\neq b \text{ con } ax=0, by=0\]
		\[ ab(x-y)=abx-aby=0-0=0\]
		\[0\neq ab \in A \Rightarrow x-y \in T(M)\]
	\end{observation}

	\begin{observation}
	\[ a \in A, x \in T(M) \Rightarrow \exists a\neq b \in A : bx=0 \Rightarrow b(ax)=0 \Rightarrow ax \in T(M) \]
	\end{observation}

	\begin{definition}
		Si $ T(M)=M $ se dice que $ M $ es un $ A $-módulo de torsión. Si $ T(M)=0 $ se dice que $ M $ es un $ A $-módulo libre de torsión.
	\end{definition}

	\begin{proposition}
		$M/T(M)$ es libre de torsión.
		\[  T(M) \hookrightarrow M \flechaSobreyectiva M/T(M)  \]
	\end{proposition}

	\begin{proof}
		\begin{flalign*}
		& \begin{rcases}
			\begin{rcases}
			\overline{x} = x + T(M) \in M/T(M) \\
			0\neq a \in A \quad a \cdot \overline{x} = \overline{ax}
			\end{rcases}\Rightarrow
		 a\cdot\overline{x}=\overline{0}  \Leftrightarrow ax\in T(M) \\
		 0\neq b\ \  b(ax)=0 \Leftrightarrow (ba)x=0, ba\neq 0
 		\end{rcases}
			\Rightarrow \overline{x} \in T(M/T(M)) \Leftrightarrow ax\in T(M) &
		\end{flalign*}
		\begin{flalign*}
		& \Leftrightarrow  x \in T(M) \Leftrightarrow \overline{x}=\overline{0} &
		\end{flalign*}
	\end{proof}

	\begin{exercise}
		$f: M \longrightarrow N$ homomorfismo de $ A $-módulo. Prooverline que $ f(T(M)) \subset T(N) $
	\end{exercise}
	\begin{definition}
		Sea $ M $ un $ A $-módulo.\\
		Una familia $ \{x_i\}_{i\in I} $ de elementos de $ M $ es linealmente independiente si:
		\[0=\sum_{i\in I}a_i x_i \  (a_i \in A, a_i =0 \text{ para casi todo }i) \Rightarrow a_i =0 \ \forall i\in I \]
	\end{definition}
	\begin{proposition}
		$ \{x_i\}_{i\in I} $ linealmente independiente $ \Leftrightarrow $ toda subfamilia finita de $ \{x_i\}_{i\in I} $ es linealmente independiente.
	\end{proposition}
	\begin{proposition}\label{prop:suma=sumadirecta}
		$ \{x_i\}_{i\in I} $ linealmente independiente.
			$ Ax_i = \langle x_i\rangle$
			\[ \oplus_{i\in I}Ax_i \xrightarrow{\ \phi\ } \sum_{i\in I} A x_i \]
			\[ (a_i x_i)_{i\in I} \longmapsto \sum_{i\in I}a_i x_i \]  es sobreyectiva e inyectiva ya que $\sum_{i\in I}a_i x_i=0\Rightarrow a_i=0 \ \forall i\in I\Rightarrow (a_i x_i)_{i\in I} = (0)_{i\in I}$.
	\end{proposition}

	\begin{example}
		\
		\begin{itemize}
			\item $ \{0\} $ no es un conjunto linealmente independiente $ a0=0, a\neq 0$.
			\item $ \{0\} \cup S$ no es linealmente independiente.	$ a0+\sum_{s\in S}0\cdot s =0 $.
		\end{itemize}
	\end{example}

	\begin{proposition}\label{prop:a=axi}
		Sea $ \{x_i\}_{i\in I} $ linealmente independiente.
		\[ \psi_i:A\longrightarrow Ax_i=\langle x_i\rangle \]
		\[ a\longmapsto \psi_i(a)=ax_i \]
		es homomorfismo de $A$-módulo sobreyectivo para todo $i$. Además,
		\[ \text{Ker}(\psi_i)=(0:x_i)=\{a\in A : ax_i=0\}=(0) \]
		Se tiene entonces:
		\[ A=A/(0:x_i)\simeq Ax_i \]
	\end{proposition}

	\begin{observation}
		$ \{x\} $ linealmente independiente $ \Leftrightarrow (0:x)=0$ 
	\end{observation} 

	\begin{definition} Una base de un $ A $-módulo $ M $ es un conjunto de generadores linealmente independiente.  \[ M=\langle \{x_i\}_{i\in I}\rangle=\sum_{i\in I} Ax_i\ \  \{x_i\}_{i\in I} \text{ linealmente independiente.} \] 
	\end{definition}
		
	\begin{definition}
		Un $ A $-módulo $ F $ es libre si tiene una base $\{x_i\}_{i\in I}$. La base cumple:
		\begin{itemize}
			\item $x\in F\Rightarrow$ x se puede expresar como $x=\sum_{i\in I}a_i x_i$.
			\item Esta expresión es única, es decir, $\sum a_i x_i=\sum b_i x_i \Rightarrow \sum (a_i-b_i)x_i=0\Rightarrow (a_i-b_i)=0 \ \forall i$
		\end{itemize}
	\end{definition}

	\begin{proposition}
		$ A $ cuerpo $ \Leftrightarrow $ todo $ A $-módulo es libre.
	\end{proposition}

	\begin{observation}
		Sea $ I $ un ideal propio no nulo de $ A $.\\
		$ M=A/I $ no es $ A $-módulo libre.
		Si $ \overline{x} \in A/I, 0\neq a\in I, 0\neq I \subset (0\overline{\lambda}), a\overline{x}=\overline{0} $
	\end{observation}
	\begin{example}

		\begin{itemize}
			\item $ A $ es $ A $-módulo libre base $ \{1\} $
			\item $ \{0\} $ es $ A $-módulo libre base $ \emptyset $
		\end{itemize}
	\end{example}
	
	\underline{\textbf{NOTACIÓN}}

	$A$ anillo, $M$ $A$-módulo e $I$ conjunto. Tenemos que:
	\begin{enumerate}
		\item $M^I= \prod_{i\in I} M_i$, donde $M_i = M$, $\forall i$.
		\item  $M^{(I)}= \oplus_{i\in I} M_i$, donde $M_i = M$, $\forall i$.
	\end{enumerate}
	Por tanto, si tenemos que $I$ es finito con $n$ elementos:
	\[M^n = M^I = M^{(I)}\]
	\begin{proposition}
		Sea $A$ un anillo y sea $I$ un conjunto. Entonces:
		\begin{center}
			$A^{(I)}$ es un $A$-módulo libre.
		\end{center}
	\end{proposition}
	\begin{proof}
		Para cada $i\in I$ tomaremos $e_i \in A^{(I)}$ definido por $e_i = (e_{ij})_{j\in I}$ con
		$e_{ij} =
		\left \{
			\begin{matrix}
			 	0, & i\neq j \\
				1, & i = j
			\end{matrix}
		\right .
		$
		Por ejemplo, $e_1 = (1,0,0)$, $e_2 = (0,1,0)$ y $e_3 = (0,0,1)$. Así pues, $\{e_i\}_{i\in I}$ es una base de $A^{(I)}$:
		\begin{enumerate}
			\item Son generadores: $\\$
			Sea $x\in A^{(I)}$, tenemos que $x = (x_i)_{i\in I}$, con casi todos los $x_i = 0$. Además:
			\[x = \sum_{i\in I} x_ie_i, \hspace{0.2cm} x_ie_i = (x_ie_{ij})_{i\in I} = (0, ..., 0, x_i, 0, ..., 0)\in A^{(I)}\]

			\item Son independientes: $\\$
			Sea $0 = \sum_{i\in I} a_ie_i$ con casi todos los $a_i = 0$, entonces $a_i = 0$, $\forall i\in I$.
			\[0 = \sum_{i\in I} a_ie_i = (a_i)_{i\in I} \Rightarrow a_i = 0, \hspace{0.2cm} \forall i\in I\]
		\end{enumerate}
	\end{proof}
	\begin{proposition}
		Sea $A$ un anillo y $F$ un $A$-módulo. Son equivalentes:

		\begin{enumerate}
			\item $F$ libre
			\item Existe un conjunto $I$ y un isomorfismo de $A$-módulos tal que:
			\[F\simeq A^{(I)}\]
		\end{enumerate}
	\end{proposition}
	\begin{proof}\ 

		$(1 \Rightarrow 2)$

		$F$ libre $\Rightarrow \exists$ base $\{x_i\}_{i\in I}$ de $F$. Por lo tanto, usando las proposiciones~\ref{prop:suma=sumadirecta} y~\ref{prop:a=axi}:

		\[F = \sum_{i\in I} Ax_i \stackrel{generadores}{=} \oplus_{i\in I} Ax_i \stackrel{lin. indep.}{=}  \oplus_{i\in I }A \stackrel{lin. indep.}{=}  A^{(I)}\]
		$(2 \Rightarrow 1)$ $\\$
		Anteriormente demostrado.
	\end{proof}
	\begin{proposition}
		Sea $F$ un $A$-módulo libre con base $\{x_i\}_{i\in I}$ y $M$ otro $A$-módulo. Además sea $\{y_i\}_{i\in I}$ un conjunto de elementos de $M$. Entonces existe un único homomorfismo de $A$-módulos de la forma:
		\[\varphi:F \longrightarrow M\]
		tal que $\varphi(x_i) = y_i$, $\forall i\in I$.
	\end{proposition}

	Ahora, tomemos $x\in F$, $x = \sum_{i\in I}a_ix_i$. Por lo que $\varphi (x) = \varphi (\sum_{i\in I}a_ix_i) = \sum_{i\in I}a_i\varphi (x_i)$. De esto obtendremos la siguiente proposición.

	\begin{proposition} Se cumple que:
		\begin{center}
			$\varphi$ es inyectivo $\Leftrightarrow$ $\{y_i\}_{i\in I}$ es lin. independiente
		\end{center}
		\begin{center}
			$\varphi$ es sobreyectivo $\Leftrightarrow$ $\{y_i\}_{i\in I}$ generadores de $M$
		\end{center}
		\begin{center}
			$\varphi$ es isomorfismo $\Leftrightarrow$ $\{y_i\}_{i\in I}$ es base de $M$
		\end{center}
	\end{proposition}

	\begin{definition}
		Sea $M$ un $A$-módulo. Un homomorfismo sobreyectivo de $A$-módulos $\epsilon : F \longrightarrow M$, con $F$ un $A$-módulo libre, se denomina \underline{presentación libre} de $M$.
	\end{definition}
	\begin{proposition}
		Todo módulo $M$ tiene una presentación libre.
	\end{proposition}
	\begin{definition}
		Sea $\{x_i\}_{i\in I}$ un conjunto de generadores de $M$ y $\epsilon : F = A^{(I)} \longrightarrow M$ una presentación libre de $M$ tal que $\epsilon (e_i) = x_i$. Se denomina módulo de relaciones de esta presentación a $N = ker(\epsilon) \subset A^{(I)}$.
	\end{definition}
	\begin{proposition}
		Sea $\{f_j\}_{j\in J}$ un conjunto de generadores de $N$. Para cada $j\in J$, $f_j = \sum_{i\in I} a_{ij}e_i$, $a_{ij}\in A$ con $a_{ij} = 0$ para casi todo $i\in I$.
	\end{proposition}
	\begin{proof}
		Para cada $j\in J$, $\epsilon (f_j) = 0 \Rightarrow 0 = \epsilon (\sum_{i\in I} a_{ij}e_i) = \sum_{i\in I} a_{ij} \epsilon (e_i) = \sum_{i\in I} a_{ij} x_i$.
	\end{proof}
	\begin{proposition}
		Sea $M$ un $A$-módulo y $\{x_i\}_{i\in I}$ un conjunto de generadores de $M$. Si además se cumple que:
		\begin{center}
			$\epsilon : A^{(I)} \longrightarrow M$, con $\epsilon (e_i) = x_i$ $\forall i\in I$  $\ $y$\ $  $0 = \sum_{i\in I} a_{ij}e_i$, $j\in J$.
		\end{center}
		Entonces $M$ es el $A$-módulo con generadores $\{x_i\}_{i\in I}$ y relaciones $\sum_{i\in I} a_{ij}e_i = 0$, $\forall j\in J$.
	\end{proposition}
	\begin{proof}
		$A^{(I)} \stackrel{\epsilon}{\longrightarrow} M$
		\[ker(\epsilon) = \langle f_j = \sum_{i\in I} a_{ij}e_i\rangle_{j\in J}\]
		para cada $j$. Así tenemos que:
		\[\frac{A^{(I)}}{ker(\epsilon)}\simeq M.\]
	\end{proof}
	%FALTA COMPLETAR UNAS COSAS
	\begin{observation}
		$A^{(I)} \longrightarrow ker(\epsilon) \longrightarrow A^{(I)} \longrightarrow M \\$
		siendo $\psi: A^{(0)} \longrightarrow A^{(I)}$ con $coker(\psi) = \frac{A^{(I)}}{Im(\psi)} = \frac{A^{(I)}}{ker(\epsilon)} \simeq M$.
	\end{observation}
	%ACABA PARTE JORGE 15 NOVIEMBRE
	\begin{proposition}
		Sea $A=\mathbb{Z}$ y $M$ $A$-módulo con generadores $x_1,x_2$ y la relación $x_1+x_2=0$. Veamos que $\mathbb{Z}\simeq M$.
	\end{proposition}

	\begin{proof}
		\ \\
		Vamos a definir un $\phi$ de forma que $\mathbb{Z}\xrightarrow{\ \phi\ }M$ sea isomorfismo.
		Definimos $\phi(a)=ax_1 \ \forall a \in \mathbb{Z}$.
		$\begin{rcases}
			x_1=\phi(1)\Rightarrow x_1\in Im(\phi) \\
			x_2=-x_1=-\phi(1)=\phi(-1)\Rightarrow x_2 \in Im(\phi)
		\end{rcases}$
		$\Rightarrow \phi$ es sobreyectiva. \\
		Veamos que $Ker(\phi)=\{0\}$. Para ello nos ayudamos del siguiente homomorfismo:
		\[ N\flechaInyectiva \mathbb{Z}^2 \xFlechaSobreyectiva{\ \pi\ } M\]
		Donde $N=\langle  e_1+e_2\rangle$, $\mathbb{Z}^2=\langle  e_1,e_2\rangle$ y definimos $\pi(e_1)=x_1, \pi(e_2)=x_2$.
		\[ \text{Sea } a \in Ker(\phi) \Rightarrow 0 = \phi(a)=ax_1=a\pi(e_1)=\pi(ae_1)\]
		\[\pi(ae_1)=0 \Rightarrow ae_1 \in Ker(\pi)=N=\langle e_1 + e_2\rangle\]
		\[\text{Luego } ae_1=b(e_1+e_2) \Rightarrow ae_1=be_1+be_2\]
		Como la expresión de $ae_1$ es única en función de elementos de la base tenemos:\\
		$\begin{rcases}
				a=b \\
				b=0
		\end{rcases}$
		$\Rightarrow a=0 \Rightarrow Ker(\phi)=\{0\}$\\
		$\phi$ es sobreyectiva e inyectiva y por lo tanto es un isomorfismo.
		\[(0)\xrightarrow{\ \ } \mathbb{Z} \xrightarrow{\ \simeq\ }M\]
	\end{proof}


	\begin{proposition}
		Sean $A$ anillo, $I$ ideal de $A$ y $M$ $A$-módulo. Se tiene:
		\begin{itemize}
			\item $M/IM$ es $A/I$-módulo.
			\item $f:M\rightarrow N$ un homomorfismo de $A$-módulos.
			\item $\bar{f}:M/IM\rightarrow N/IN$ homomorfismo de $A/I$-módulos.
			\item $f$ es isomorfismo $\Rightarrow \bar{f}$ es isomorfismo ($f(IM)=IN$).
		\end{itemize}
	\end{proposition}
	
	\begin{proof} 
		Veamos que $\bar{f}$ es isomorfismo.

		$\sum a_iy_i \in IN\ $ y $\  y_i=f(x_i)  \ \ x_i\in M \Rightarrow \sum a_iy_i=\sum a_if(x_i)=f(\sum a_ix_i)\in f(IM)$

		$f(IM) \ni f(\sum a_ix_i) = \sum a_if(x_i) \ $ y $\ f(x_i)\in N \Rightarrow f(\sum a_ix_i)\in IN$

		Tenemos demostrado que $f(IM)=IN$. A partir de esto se deduce que $\bar{f}$ homomorfismo bien definido y de $f$ sobreyectiva se deduce $\bar{f}$ sobreyectiva. Veamos que $\bar{f}$ es inyectiva.

		$\bar{f}(\overline{x})=0 \Leftrightarrow \overline{f(x)} = IN \Leftrightarrow f(x) \in IN=f(IM) \Leftrightarrow x \in IM \Leftrightarrow \overline{x}=IM=0$

	\end{proof}

	\begin{lemma}
		Sea $A$ un anillo, $X$ un conjunto e $I$ un ideal de $A$. Se tiene un isomorfismo de $A/I$-módulos.
		\[A^{(X)}/IA^{(X)} \xrightarrow{\ \simeq\ }(A/I)^{(X)}\]
	\end{lemma}

	\begin{proof}

		$A\rightarrow A/I$ homomorfismo de $A$-módulos definido como $a \leadsto \overline{a}$.

		$A^{(X)} \xrightarrow{\ \phi\ }(A/I)^{(X)}$ tal que $(a_i)\leadsto (\overline{a}_i)_{i \in X}.$

		$\phi$ es un homomorfismo de $A$-módulos sobreyectivo.

		$Ker(\phi)=\{(a_i)_{i\in X} \in A^{(X)} \ / \ a_i \in I \ \forall i \in X\} = IA^{(X)}.$

		($\subset$)

		Sea $(a_i)_{i\in X} \in A^{(X)} \ / \ a_i \in X \ \forall i \in X$

		$(a_i)=\sum_{i \in X}a_ie_i \in IA^{(X)}$ \ \ ($\{e_i\}_{i\in X}$ base canónica de $A^{(X)}$)

		($\supset$)

		Sea $\sum_{j=1}^nb_jf_j \ \ b_j \in I, \ f_j=(f_{ji})_{i\in X} \in A^{(X)}$

		$b_jf_j=b(f_{ji})_{i\in X} = (bf_{ji})_{i\in X} \in Ker(\phi) \text{ ya que }bf_{ji} \in I \Rightarrow \sum_{j=1}^nb_jf_j \in Ker(\phi)$.

		\[A^{(X)}/IA^{(X)} \xrightarrow{\ \simeq \ } (A/I)^{(X)} \]
	\end{proof}

	\begin{observation}
		\[A\rightarrow A/I \ \  M,N \  A/I\text{-módulos}\]
		\[f:M\rightarrow N \text{ es homomorfismo de $A$-módulos} \Rightarrow \text{es homomorfismo de $A/I$-módulos}\]
		\[IM=0=IN\]
	\end{observation}

	\begin{theorem}
		Dos bases de un $A$-módulo libre tienen el mismo cardinal.
	\end{theorem}

	\begin{proof}
		\[A^{(I)} \simeq F \simeq A^{(J)}\]
		\[\exists \  A^{(I)} \rightarrow A^{(J)} \text{ isomorfismo}\]
		Sea $m$ un ideal maximal de $A$. Aplicando el lema:
		\[ (A/m)^{(I)} = A^{(I)}/mA^{(I)} \simeq A^{(J)}/mA^{(J)} = (A/m)^{(J)} \]
		$A/m$ es un cuerpo. Luego tenemos un isomorfismo de espacios vectoriales.
		\[dim((A/m)^{(I)})=card(I) \text{ y } dim((A/m)^{(J)})=card(J)\]
		Conclusión $card(I)=card(J)$.
	\end{proof}

	\begin{definition}
		El cardinal de una base de un $A$-módulo libre se llama rango del módulo libre.
	\end{definition}

\end{document}
