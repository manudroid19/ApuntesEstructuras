

%-----------------------------------------------------------------------------------------------------
%	INCLUSIÓN DE PAQUETES BÁSICOS
%-----------------------------------------------------------------------------------------------------

\documentclass{article}
\usepackage{enumerate}
\usepackage{wrapfig}
\usepackage{graphicx}



%-----------------------------------------------------------------------------------------------------
%	SELECCIÓN DEL LENGUAJE
%-----------------------------------------------------------------------------------------------------

% Paquetes para adaptar Látex al Español:
\usepackage[spanish,es-noquoting, es-tabla, es-lcroman]{babel} % Cambia
\usepackage[utf8]{inputenc}                                    % Permite los acentos.
\selectlanguage{spanish}                                       % Selecciono como lenguaje el Español.

%flechassssss
\def\flechaSobreyectiva{\mathrel{\mkern16mu  \vcenter{\hbox{$\scriptscriptstyle+$}}%
		\mkern-25mu{\longrightarrow}}}
\def\flechaInyectiva{\mathrel{\mkern0mu  \vcenter{\hbox{$\scriptscriptstyle+$}}%
		\mkern-9mu{\longrightarrow}}}
\def\flechaBiyectiva{\mathrel{\mkern16mu  \vcenter{\hbox{$\scriptscriptstyle+$}}\mkern-25mu{\mathrel{\mkern0mu\vcenter{\hbox{$\scriptscriptstyle+$}}\mkern-9mu{\longrightarrow}}}}}
\def\xFlechaInyectiva #1{\mathrel{\ooalign{\thinspace\thinspace$\mapstochar\mkern5mu$\hfil\cr$\xrightarrow{#1}$\cr}}}
\def\xFlechaSobreyectiva #1{\mathrel{\ooalign{\hfil$\mapstochar\mkern5mu$\thinspace\thinspace\cr$\xrightarrow{#1}$\cr}}}
%llave a la derecha
\newenvironment{rcases}
{\left.\begin{aligned}}
	{\end{aligned}\right\rbrace}

%-----------------------------------------------------------------------------------------------------
%	SELECCIÓN DE LA FUENTE
%-----------------------------------------------------------------------------------------------------

% Fuente utilizada.
\usepackage{courier}                    % Fuente Courier.
\usepackage{microtype}                  % Mejora la letra final de cara al lector.

%-----------------------------------------------------------------------------------------------------
%	ESTILO DE PÁGINA
%-----------------------------------------------------------------------------------------------------

% Paquetes para el diseño de página:
\usepackage{fancyhdr}               % Utilizado para hacer títulos propios.
\usepackage{lastpage}               % Referencia a la última página. Utilizado para el pie de página.
\usepackage{extramarks}             % Marcas extras. Utilizado en pie de página y cabecera.
\usepackage[parfill]{parskip}       % Crea una nueva línea entre párrafos.
\usepackage{geometry}               % Asigna la "geometría" de las páginas.

% Se elige el estilo fancy y márgenes de 3 centímetros.
\pagestyle{fancy}
\geometry{left=3cm,right=3cm,top=3cm,bottom=3cm,headheight=1cm,headsep=0.5cm} % Márgenes y cabecera.
% Se limpia la cabecera y el pie de página para poder rehacerlos luego.
\fancyhf{}

% Espacios en el documento:
\linespread{1.1}                        % Espacio entre líneas.
\setlength\parindent{0pt}               % Selecciona la indentación para cada inicio de párrafo.

% Cabecera del documento. Se ajusta la línea de la cabecera.
\renewcommand\headrule{
	\begin{minipage}{1\textwidth}
		\hrule width \hsize
	\end{minipage}
}

% Texto de la cabecera:
\lhead{\docauthor}                          % Parte izquierda.
\chead{}                                    % Centro.
\rhead{\subject \ - \doctitle}              % Parte derecha.

% Pie de página del documento. Se ajusta la línea del pie de página.
\renewcommand\footrule{
	\begin{minipage}{1\textwidth}
		\hrule width \hsize
	\end{minipage}\par
}

\lfoot{}                                                 % Parte izquierda.
\cfoot{}                                                 % Centro.
\rfoot{Página\ \thepage\ de\ \protect\pageref{LastPage}} % Parte derecha.

%----------------------------------------------------------------------------------------
%	MATEMÁTICAS
%----------------------------------------------------------------------------------------

% Paquetes para matemáticas:
\usepackage{amsmath, amsthm, amssymb, amsfonts, amscd} % Teoremas, fuentes y símbolos.

% Nuevo estilo para definiciones
\newtheoremstyle{definition-style} % Nombre del estilo
{5pt}                % Espacio por encima
{5pt}                % Espacio por debajo
{}                   % Fuente del cuerpo
{}                   % Identación: vacío= sin identación, \parindent = identación del parráfo
{\bf}                % Fuente para la cabecera
{.}                  % Puntuación tras la cabecera
{.5em}               % Espacio tras la cabecera: { } = espacio usal entre palabras, \newline = nueva línea
{}                   % Especificación de la cabecera (si se deja vaía implica 'normal')

% Nuevo estilo para teoremas
\newtheoremstyle{theorem-style} % Nombre del estilo
{5pt}                % Espacio por encima
{5pt}                % Espacio por debajo
{\itshape}           % Fuente del cuerpo
{}                   % Identación: vacío= sin identación, \parindent = identación del parráfo
{\bf}                % Fuente para la cabecera
{.}                  % Puntuación tras la cabecera
{.5em}               % Espacio tras la cabecera: { } = espacio usal entre palabras, \newline = nueva línea
{}                   % Especificación de la cabecera (si se deja vaía implica 'normal')

% Nuevo estilo para ejemplos y ejercicios
\newtheoremstyle{example-style} % Nombre del estilo
{5pt}                % Espacio por encima
{5pt}                % Espacio por debajo
{}                   % Fuente del cuerpo
{}                   % Identación: vacío= sin identación, \parindent = identación del parráfo
{\scshape}                % Fuente para la cabecera
{:}                  % Puntuación tras la cabecera
{.5em}               % Espacio tras la cabecera: { } = espacio usal entre palabras, \newline = nueva línea
{}                   % Especificación de la cabecera (si se deja vaía implica 'normal')

% Teoremas:
\theoremstyle{theorem-style}  % Otras posibilidades: plain (por defecto), definition, remark
\newtheorem{theorem}{Teorema}[section]  % [section] indica que el contador se reinicia cada sección
\newtheorem{corollary}[theorem]{Corolario} % [theorem] indica que comparte el contador con theorem
\newtheorem{lemma}[theorem]{Lema}
\newtheorem{proposition}[theorem]{Proposición}

% Definiciones, notas, conjeturas
\theoremstyle{definition}
\newtheorem{definition}{Definición}[section]
\newtheorem{conjecture}{Conjetura}[section]
\newtheorem*{note}{Nota} % * indica que no tiene contador
\newtheorem*{observation}{Observación} % * indica que no tiene contador
\newtheorem*{properties}{Propiedades}
\newtheorem*{comment}{Comentario clase}

% Ejemplos, ejercicios
\theoremstyle{example-style}
\newtheorem{example}{Ejemplo}[section]
\newtheorem{exercise}{Ejercicio}[section]

%-----------------------------------------------------------------------------------------------------
%	PORTADA
%-----------------------------------------------------------------------------------------------------

% Elija uno de los siguientes formatos.
% No olvide incluir los archivos .sty asociados en el directorio del documento.
%\usepackage{title1}
\usepackage{title2}
%\usepackage{title3}

%-----------------------------------------------------------------------------------------------------
%	TÍTULO, AUTOR Y OTROS DATOS DEL DOCUMENTO
%-----------------------------------------------------------------------------------------------------

% Título del documento.
\newcommand{\doctitle}{Tema 4}
% Subtítulo.
\newcommand{\docsubtitle}{Teorema de estructura de módulos.}
% Fecha.
\newcommand{\docdate}{20 \ de \ Noviembre \ de \ 2018}
% Asignatura.
\newcommand{\subject}{Estructuras algebraicas}
% Autor.
\newcommand{\docauthor}{Sergio Mayo, Manuel de Prada, Jorge Vázquez}
\newcommand{\docaddress}{Universidade de Santiago de Compostela}
\newcommand{\docemail}{lalala@gmail.com}

%-----------------------------------------------------------------------------------------------------
%	RESUMEN
%-------------------------------					----------------------------------------------------------------------

% Resumen del documento. Va en la portada.
% Puedes también dejarlo vacío, en cuyo caso no aparece en la portada.
%\newcommand{\docabstract}{}
\newcommand{\docabstract}{Tema de módulos del peazo profe Javier Majaderias.}

\begin{document}

\maketitle

%-----------------------------------------------------------------------------------------------------
%	ÍNDICE
%-----------------------------------------------------------------------------------------------------

% Profundidad del Índice:
%\setcounter{tocdepth}{1}

\newpage
\tableofcontents
\newpage

%----------------------------------------------------------------------------------------
%	Sección 1: Deficiones y teoremas
%----------------------------------------------------------------------------------------

\section{Módulos de tipo finito sobre un DIP}
	\subsection{Equivalencia de matrices.}
	En esta sección, $ R $ denota un anillo (conmutativo). 
	\begin{lemma}
		Sea $ E_{ij} $ la matriz cuadrada sobre $ R $ con 1 en el lugar $ (i,j) $ y 0 en todos los demás lugares. Se verifica:
		\begin{enumerate}[\hspace{1cm}i)]
			\item $E_{ij} \cdot E_{rs}= \begin{cases}
			0 &\text{si } j\neq r\\
			E_{is} &\text{si } j=r
			\end{cases}$
			\item Para cada matriz $A$, se tiene:\\
				$ E_{ij} \cdot A= $ matriz que hace que todas las filas nulas salvo la $ i $-ésima, en la cual aparece la $ j $-ésima de $ A $.
				
				$A \cdot E_{ij} = $ matriz que hace que todas las columnas nulas salvo la $ j $-ésima, en la cual aparece la $ i $-ésima de $ A $.
		\end{enumerate}
	\end{lemma}
	\begin{definition}
		Se llaman matrices elementales a las matrices de cada uno de los siguientes tres tipos:
		\begin{enumerate}[\hspace{1cm}I)]
			\item Sea $ b\in R, i\neq j :$
			\[ T_{ij}(b) := I+b\cdot E_{ij}\]
			$ T_{ij}(b)= $ matriz cuadrada con 1 en la diagonal, $ b $ en el lugar $ r_{ij} $ y 0 en todos los demás lugares.
			
			$ T_{ij}(b) $ es inversible, siendo $ T_{ij}(b)^{-1}=T_{ij}(-b) $. Teniendo en cuenta que $E_{ij}\cdot E_{ij}=0$ si $j\neq i$, comprobamos la inversibilidad:
			\[(I+bE_{ij})(I-bE_{ij})=I-bE_{ij}+bE_{ij}-b^2E_{ij}E_{ij}=I\]
			\item Sea $ u $ unidad de $ R $:
			\[ D_i(u):=I+(u-1) E_{ii} \]
			$ D_i(u) = $ matriz diagonal con u en el lugar (i,i) y 1 en los demás lugares de la diagonal $( D_i(u):=I+uE_{ii}-E_{ii})$.
			
			$ D_i(u)$ es inversible, con $ D_i(u)^{-1}=D_i(u^{-1}) $. Veámoslo ($E_{ii}\cdot E_{ii}=E_{ii}$):
			\[ [I+(u-1) E_{ii}][I+(u^{-1}-1) E_{ii}] = I+(u^{-1}-1)E_{ii}+(u-1)E_{ii}+(u^{-1}-1)(u-1)E_{ii}E_{ii}=\]
			\[=I+u^{-1}E_{ii}-E_{ii}+uE_{ii}-E_{ii}+(1-u^{-1}-u+1)E_{ii}=I\]
			\item $ P_{ij}:=I-E_{ii}-E_{jj}+E_{ij}+E_{ji}= \begin{bmatrix}
			1 & \cdots & 0 & \cdots & 0 & \cdots & 0  \\
			\vdots & \ddots &\vdots & &\vdots & &\vdots \\
			0 & \cdots & 0 & \cdots & 1 & \cdots & 0  \\
			\vdots & &\vdots & \ddots &\vdots & &\vdots \\
			0 & \cdots & 1 & \cdots & 0 & \cdots & 0  \\
			\vdots & &\vdots & &\vdots & \ddots &\vdots \\
			0 & \cdots & 0 & \cdots & 0 & \cdots & 0  \\
			\end{bmatrix} $
			
			$ P_{ij}  $ es inversible: $ P_{ij}^{-1}=P_{ij} $. Comprobémoslo:
			
			$ P_{ij}\cdot P_{ij}=I-E_{ii}-E_{jj}+E_{ij}+E_{ji}-E_{ii}+(E_{ii}E_{ii}=E_{ii})+(E_{ii}E_{jj}=0)-(E_{ii}E_{ij}=E_{ij})-(E_{ii}E_{ji}=0)-E_{jj} +(E_{jj}E_{ii}=0)+(E_{jj}E_{jj}=E_{jj})-(E_{jj}E_{ij}=0)-(E_{jj}E_{ji}=E_{ji})+E_{ij}-(E_{ij}E_{ii}=0)-(E_{ij}E_{jj}=E_{ij})+(E_{ij}E_{ij}=0)+(E_{ij}E_{ji}=E_{ii})+E_{jj}-(E_{ji}E_{ii}=E_{ji})-(E_{ji}E_{jj}=0)+(E_{ji}E_{ij}=E_{jj})+(E_{ji}E_{ji})=I$.
			
		\end{enumerate}
	\end{definition}
	\begin{definition}
		Si $ A\in M_{m \times n}(R) $, se llaman operaciones elementales en $ A $, la multiplicación de $ A $ por la izquierda o por la derecha por matrices elementales. 
		Las operaciones elementales son, por tanto:
		\begin{enumerate}
			\item $ T_{ij}(b) \cdot A = $ matriz obtenida de $ A $ sumándole a la fila $ i $ la $ j $ multiplicada por $ b $. 
			
			$T_{ij}(b) \cdot A=(I+bE_{ij})A=A+bE_{ij}A$
			\item $ A \cdot T_{ij}(b) $ matriz obtenida de $ A $ sumándole a la columna $ j $ la $ i $ multiplicada por $ b $.
			\item $ D_i(u) \cdot A =$ matriz obtenida de $ A $ multiplicando la fila $ i $ por $ u $.
			
			$ D_i(u)\cdot A= (I+(u-1)E_{ii})A=A+(u-1) E_{ii}A=A-E_{i_i} A +uE_{ii}A $
			\item $ A \cdot D_i(u)  =$ matriz obtenida de $ A $ multiplicando la columna $ i $ por $ u $.
			\item $ P_{ij} \cdot A = $ matriz obtenida a partir de A intercambiando las filas $ i $ y $ j $.
			
			$ P_{ij}A= (I-E_{ii}-E_{jj}+E_{ij}+E_{ji})A=A-E_{ii}A-E_{jj}A+E_{ij}A+E_{ji}A $  
			\item $ A \cdot P_{ij}= $ matriz obtenida a partir de A intercambiando las columnas $ i $ y $ j $.
		\end{enumerate}
	\end{definition}
	\begin{definition}
		Dos matrices $ A, B \in M_{m\times n}(R)  $se dicen equivalentes si existen matrices $ Q \in M_{m\times m}(R), P \in M_{n\times n}(R) $ tales que: 
		\[ B=QAP \]
		Nótese que las matrices obtenidas de $ A $ por un número finito de operaciones elementales son equivalentes a $ A $.
		\[ U_q \cdots U_1AV_1\cdots V_p=QAP\]
	\end{definition}
	\section{Diagonalización}
	\begin{theorem}
		Sea $ D $ un dominio de ideales principales. Si $ A \in M_{m\times n}(D) $, entonces $ A $ es equivalente a una matriz diagonal.
		\[ \text{diag}(d_1, \cdots, d_r, 0, \cdots, 0), \quad \text{con } 0\leq r \leq \min \{m,n\} \text{ y }\begin{cases}
			d_i\neq 0\\
			d_i \mid d_j $ si $ i\leq j
		\end{cases}\]
		
		
	\end{theorem}
	\begin{example}
		Una matriz $3\times 2$ de diagonal $(d_1,d_2)$ sería $\begin{bmatrix}
		d_1 & 0  \\
		0   & d_2\\
		0   & 0  \\
		\end{bmatrix}$
	\end{example}
	\begin{observation}
		Si el DIP es euclídeo la diagonalización se realiza mediante operaciones elementales, si no lo es, necesitaremos otras operaciones.
	\end{observation}
	\begin{proof}
		
	\end{proof}
	
\end{document}
