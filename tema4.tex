

%-----------------------------------------------------------------------------------------------------
%	INCLUSIÓN DE PAQUETES BÁSICOS
%-----------------------------------------------------------------------------------------------------

\documentclass{article}
\usepackage{enumerate}
\usepackage{wrapfig}
\usepackage{graphicx}



%-----------------------------------------------------------------------------------------------------
%	SELECCIÓN DEL LENGUAJE
%-----------------------------------------------------------------------------------------------------

% Paquetes para adaptar Látex al Español:
\usepackage[spanish,es-noquoting, es-tabla, es-lcroman]{babel} % Cambia
\usepackage[utf8]{inputenc}                                    % Permite los acentos.
\selectlanguage{spanish}                                       % Selecciono como lenguaje el Español.

%flechassssss
\def\flechaBiyectiva{\mathrel{\mkern16mu  \vcenter{\hbox{$\scriptscriptstyle+$}}\mkern-25mu{\mathrel{\mkern0mu\vcenter{\hbox{$\scriptscriptstyle+$}}\mkern-9mu{\longrightarrow}}}}}
\def\xFlechaInyectiva #1{\mathrel{\ooalign{\thinspace\thinspace$\mapstochar\mkern5mu$\hfil\cr$\xrightarrow{#1}$\cr}}}
\def\flechaInyectiva {\mathrel{\ooalign{\thinspace\thinspace$\mapstochar\mkern5mu$\hfil\cr$\longrightarrow$\cr}}}
\def\xFlechaSobreyectiva #1{\mathrel{\ooalign{\hfil$\mapstochar\mkern5mu$\thinspace\thinspace\cr$\xrightarrow{#1}$\cr}}}
\def\flechaSobreyectiva {\mathrel{\ooalign{\hfil$\mapstochar\mkern5mu$\thinspace\thinspace\cr$\longrightarrow$\cr}}}
%llave a la derecha
\newenvironment{rcases}
{\left.\begin{aligned}}
	{\end{aligned}\right\rbrace}

%-----------------------------------------------------------------------------------------------------
%	SELECCIÓN DE LA FUENTE
%-----------------------------------------------------------------------------------------------------

% Fuente utilizada.
\usepackage{courier}                    % Fuente Courier.
\usepackage{microtype}                  % Mejora la letra final de cara al lector.

%-----------------------------------------------------------------------------------------------------
%	ESTILO DE PÁGINA
%-----------------------------------------------------------------------------------------------------

% Paquetes para el diseño de página:
\usepackage{fancyhdr}               % Utilizado para hacer títulos propios.
\usepackage{lastpage}               % Referencia a la última página. Utilizado para el pie de página.
\usepackage{extramarks}             % Marcas extras. Utilizado en pie de página y cabecera.
\usepackage[parfill]{parskip}       % Crea una nueva línea entre párrafos.
\usepackage{geometry}               % Asigna la "geometría" de las páginas.

% Se elige el estilo fancy y márgenes de 3 centímetros.
\pagestyle{fancy}
\geometry{left=3cm,right=3cm,top=3cm,bottom=3cm,headheight=1cm,headsep=0.5cm} % Márgenes y cabecera.
% Se limpia la cabecera y el pie de página para poder rehacerlos luego.
\fancyhf{}

% Espacios en el documento:
\linespread{1.1}                        % Espacio entre líneas.
\setlength\parindent{0pt}               % Selecciona la indentación para cada inicio de párrafo.

% Cabecera del documento. Se ajusta la línea de la cabecera.
\renewcommand\headrule{
	\begin{minipage}{1\textwidth}
		\hrule width \hsize
	\end{minipage}
}

% Texto de la cabecera:
\lhead{\docauthor}                          % Parte izquierda.
\chead{}                                    % Centro.
\rhead{\subject \ - \doctitle}              % Parte derecha.

% Pie de página del documento. Se ajusta la línea del pie de página.
\renewcommand\footrule{
	\begin{minipage}{1\textwidth}
		\hrule width \hsize
	\end{minipage}\par
}

\lfoot{}                                                 % Parte izquierda.
\cfoot{}                                                 % Centro.
\rfoot{Página\ \thepage\ de\ \protect\pageref{LastPage}} % Parte derecha.

%----------------------------------------------------------------------------------------
%	MATEMÁTICAS
%----------------------------------------------------------------------------------------

% Paquetes para matemáticas:
\usepackage{amsmath, amsthm, amssymb, amsfonts, amscd} % Teoremas, fuentes y símbolos.

% Nuevo estilo para definiciones
\newtheoremstyle{definition-style} % Nombre del estilo
{5pt}                % Espacio por encima
{5pt}                % Espacio por debajo
{}                   % Fuente del cuerpo
{}                   % Identación: vacío= sin identación, \parindent = identación del parráfo
{\bf}                % Fuente para la cabecera
{.}                  % Puntuación tras la cabecera
{.5em}               % Espacio tras la cabecera: { } = espacio usal entre palabras, \newline = nueva línea
{}                   % Especificación de la cabecera (si se deja vaía implica 'normal')

% Nuevo estilo para teoremas
\newtheoremstyle{theorem-style} % Nombre del estilo
{5pt}                % Espacio por encima
{5pt}                % Espacio por debajo
{\itshape}           % Fuente del cuerpo
{}                   % Identación: vacío= sin identación, \parindent = identación del parráfo
{\bf}                % Fuente para la cabecera
{.}                  % Puntuación tras la cabecera
{.5em}               % Espacio tras la cabecera: { } = espacio usal entre palabras, \newline = nueva línea
{}                   % Especificación de la cabecera (si se deja vaía implica 'normal')

% Nuevo estilo para ejemplos y ejercicios
\newtheoremstyle{example-style} % Nombre del estilo
{5pt}                % Espacio por encima
{5pt}                % Espacio por debajo
{}                   % Fuente del cuerpo
{}                   % Identación: vacío= sin identación, \parindent = identación del parráfo
{\scshape}                % Fuente para la cabecera
{:}                  % Puntuación tras la cabecera
{.5em}               % Espacio tras la cabecera: { } = espacio usal entre palabras, \newline = nueva línea
{}                   % Especificación de la cabecera (si se deja vaía implica 'normal')

% Teoremas:
\theoremstyle{theorem-style}  % Otras posibilidades: plain (por defecto), definition, remark
\newtheorem{theorem}{Teorema}[section]  % [section] indica que el contador se reinicia cada sección
\newtheorem{corollary}[theorem]{Corolario} % [theorem] indica que comparte el contador con theorem
\newtheorem{lemma}[theorem]{Lema}
\newtheorem{proposition}[theorem]{Proposición}

% Definiciones, notas, conjeturas
\theoremstyle{definition}
\newtheorem{definition}{Definición}[section]
\newtheorem{conjecture}{Conjetura}[section]
\newtheorem*{note}{Nota} % * indica que no tiene contador
\newtheorem*{observation}{Observación} % * indica que no tiene contador
\newtheorem*{properties}{Propiedades}
\newtheorem*{comment}{Comentario clase}

% Ejemplos, ejercicios
\theoremstyle{example-style}
\newtheorem{example}{Ejemplo}[section]
\newtheorem{exercise}{Ejercicio}[section]

%-----------------------------------------------------------------------------------------------------
%	PORTADA
%-----------------------------------------------------------------------------------------------------

% Elija uno de los siguientes formatos.
% No olvide incluir los archivos .sty asociados en el directorio del documento.
%\usepackage{title1}
\usepackage{title2}
%\usepackage{title3}

%-----------------------------------------------------------------------------------------------------
%	TÍTULO, AUTOR Y OTROS DATOS DEL DOCUMENTO
%-----------------------------------------------------------------------------------------------------

% Título del documento.
\newcommand{\doctitle}{Tema 4}
% Subtítulo.
\newcommand{\docsubtitle}{Teorema de estructura de módulos.}
% Fecha.
\newcommand{\docdate}{20 \ de \ Noviembre \ de \ 2018}
% Asignatura.
\newcommand{\subject}{Estructuras algebraicas}
% Autor.
\newcommand{\docauthor}{Sergio Mayo, Manuel de Prada, Jorge Vázquez}
\newcommand{\docaddress}{Universidade de Santiago de Compostela}
\newcommand{\docemail}{lalala@gmail.com}

%-----------------------------------------------------------------------------------------------------
%	RESUMEN
%-------------------------------					----------------------------------------------------------------------

% Resumen del documento. Va en la portada.
% Puedes también dejarlo vacío, en cuyo caso no aparece en la portada.
%\newcommand{\docabstract}{}
\newcommand{\docabstract}{Tema de módulos del peazo profe Javier Majaderias.}

\begin{document}

\maketitle

%-----------------------------------------------------------------------------------------------------
%	ÍNDICE
%-----------------------------------------------------------------------------------------------------

% Profundidad del Índice:
%\setcounter{tocdepth}{1}

\newpage
\tableofcontents
\newpage

%----------------------------------------------------------------------------------------
%	Sección 1: Deficiones y teoremas
%----------------------------------------------------------------------------------------

\section{Módulos de tipo finito sobre un DIP}
	\subsection{Equivalencia de matrices.}
	En esta sección, $ R $ denota un anillo (conmutativo). 
	\begin{lemma}
		Sea $ E_{ij} $ la matriz cuadrada sobre $ R $ con 1 en el lugar $ (i,j) $ y 0 en todos los demás lugares. Se verifica:
		\begin{enumerate}[\hspace{1cm}i)]
			\item $E_{ij} \cdot E_{rs}= \begin{cases}
			0 &\text{si } j\neq r\\
			E_{is} &\text{si } j=r
			\end{cases}$
			\item Para cada matriz $A$, se tiene:\\
				$ E_{ij} \cdot A= $ matriz que hace que todas las filas nulas salvo la $ i $-ésima, en la cual aparece la $ j $-ésima de $ A $.
				
				$A \cdot E_{ij} = $ matriz que hace que todas las columnas nulas salvo la $ j $-ésima, en la cual aparece la $ i $-ésima de $ A $.
		\end{enumerate}
	\end{lemma}
	\begin{definition}
		Se llaman matrices elementales a las matrices de cada uno de los siguientes tres tipos:
		\begin{enumerate}[\hspace{1cm}I)]
			\item Sea $ b\in R, i\neq j :$
			\[ T_{ij}(b) := I+b\cdot E_{ij}\]
			$ T_{ij}(b)= $ matriz cuadrada con 1 en la diagonal, $ b $ en el lugar $ r_{ij} $ y 0 en todos los demás lugares.
			
			$ T_{ij}(b) $ es inversible, siendo $ T_{ij}(b)^{-1}=T_{ij}(-b) $. Teniendo en cuenta que $E_{ij}\cdot E_{ij}=0$ si $j\neq i$, comprobamos la inversibilidad:
			\[(I+bE_{ij})(I-bE_{ij})=I-bE_{ij}+bE_{ij}-b^2E_{ij}E_{ij}=I\]
			\item Sea $ u $ unidad de $ R $:
			\[ D_i(u):=I+(u-1) E_{ii} \]
			$ D_i(u) = $ matriz diagonal con u en el lugar (i,i) y 1 en los demás lugares de la diagonal $( D_i(u):=I+uE_{ii}-E_{ii})$.
			
			$ D_i(u)$ es inversible, con $ D_i(u)^{-1}=D_i(u^{-1}) $. Veámoslo ($E_{ii}\cdot E_{ii}=E_{ii}$):
			\[ [I+(u-1) E_{ii}][I+(u^{-1}-1) E_{ii}] = I+(u^{-1}-1)E_{ii}+(u-1)E_{ii}+(u^{-1}-1)(u-1)E_{ii}E_{ii}=\]
			\[=I+u^{-1}E_{ii}-E_{ii}+uE_{ii}-E_{ii}+(1-u^{-1}-u+1)E_{ii}=I\]
			\item $ P_{ij}:=I-E_{ii}-E_{jj}+E_{ij}+E_{ji}= \begin{bmatrix}
			1 & \cdots & 0 & \cdots & 0 & \cdots & 0  \\
			\vdots & \ddots &\vdots & &\vdots & &\vdots \\
			0 & \cdots & 0 & \cdots & 1 & \cdots & 0  \\
			\vdots & &\vdots & \ddots &\vdots & &\vdots \\
			0 & \cdots & 1 & \cdots & 0 & \cdots & 0  \\
			\vdots & &\vdots & &\vdots & \ddots &\vdots \\
			0 & \cdots & 0 & \cdots & 0 & \cdots & 0  \\
			\end{bmatrix} $
			
			$ P_{ij}  $ es inversible: $ P_{ij}^{-1}=P_{ij} $. Comprobémoslo:
			
			$ P_{ij}\cdot P_{ij}=I-E_{ii}-E_{jj}+E_{ij}+E_{ji}-E_{ii}+(E_{ii}E_{ii}=E_{ii})+(E_{ii}E_{jj}=0)-(E_{ii}E_{ij}=E_{ij})-(E_{ii}E_{ji}=0)-E_{jj} +(E_{jj}E_{ii}=0)+(E_{jj}E_{jj}=E_{jj})-(E_{jj}E_{ij}=0)-(E_{jj}E_{ji}=E_{ji})+E_{ij}-(E_{ij}E_{ii}=0)-(E_{ij}E_{jj}=E_{ij})+(E_{ij}E_{ij}=0)+(E_{ij}E_{ji}=E_{ii})+E_{jj}-(E_{ji}E_{ii}=E_{ji})-(E_{ji}E_{jj}=0)+(E_{ji}E_{ij}=E_{jj})+(E_{ji}E_{ji})=I$.
			
		\end{enumerate}
	\end{definition}
	\begin{definition}
		Si $ A\in M_{m \times n}(R) $, se llaman operaciones elementales en $ A $, la multiplicación de $ A $ por la izquierda o por la derecha por matrices elementales. 
		Las operaciones elementales son, por tanto:
		\begin{enumerate}
			\item $ T_{ij}(b) \cdot A = $ matriz obtenida de $ A $ sumándole a la fila $ i $ la $ j $ multiplicada por $ b $. 
			
			$T_{ij}(b) \cdot A=(I+bE_{ij})A=A+bE_{ij}A$
			\item $ A \cdot T_{ij}(b) $ matriz obtenida de $ A $ sumándole a la columna $ j $ la $ i $ multiplicada por $ b $.
			\item $ D_i(u) \cdot A =$ matriz obtenida de $ A $ multiplicando la fila $ i $ por $ u $.
			
			$ D_i(u)\cdot A= (I+(u-1)E_{ii})A=A+(u-1) E_{ii}A=A-E_{i_i} A +uE_{ii}A $
			\item $ A \cdot D_i(u)  =$ matriz obtenida de $ A $ multiplicando la columna $ i $ por $ u $.
			\item $ P_{ij} \cdot A = $ matriz obtenida a partir de A intercambiando las filas $ i $ y $ j $.
			
			$ P_{ij}A= (I-E_{ii}-E_{jj}+E_{ij}+E_{ji})A=A-E_{ii}A-E_{jj}A+E_{ij}A+E_{ji}A $  
			\item $ A \cdot P_{ij}= $ matriz obtenida a partir de A intercambiando las columnas $ i $ y $ j $.
		\end{enumerate}
	\end{definition}
	\begin{definition}
		Dos matrices $ A, B \in M_{m\times n}(R)  $se dicen equivalentes si existen matrices $ Q \in M_{m\times m}(R), P \in M_{n\times n}(R) $ tales que: 
		\[ B=QAP \]
		Nótese que las matrices obtenidas de $ A $ por un número finito de operaciones elementales son equivalentes a $ A $.
		\[ U_q \cdots U_1AV_1\cdots V_p=QAP\]
	\end{definition}
	\section{Diagonalización}
	\begin{theorem}
		Sea $ D $ un dominio de ideales principales. Si $ A \in M_{m\times n}(D) $, entonces $ A $ es equivalente a una matriz diagonal.
		\[ \text{diag}(d_1, \cdots, d_r, 0, \cdots, 0), \quad \text{con } 0\leq r \leq \min \{m,n\} \text{ y }\begin{cases}
			d_i\neq 0\\
			d_i \mid d_j $ si $ i\leq j
		\end{cases}\]
		
		
	\end{theorem}
	\begin{example}
		Una matriz $3\times 2$ de diagonal $(d_1,d_2)$ sería $\begin{bmatrix}
		d_1 & 0  \\
		0   & d_2\\
		0   & 0  \\
		\end{bmatrix}$
	\end{example}
	\begin{observation}
		Si el DIP es euclídeo la diagonalización se realiza mediante operaciones elementales, si no lo es, necesitaremos otras operaciones.
	\end{observation}
	\begin{proof}
		
	\end{proof}

BEGIN SERGIO 26-11


END SERGIO 26-11

	PRADA 27-11
	\begin{theorem}
		Sea $ D $ un DIP y $ M $ un $ D $-módulo de tipo finito. Entonces $ M $ es suma directa de $ D$-módulos cíclicos.
		\[ M=Dz_1 \oplus \cdots \oplus Dz_s \]
		tal que $ (0:z_1) \supset (0:z_2) \supset ... \supset (0:z_s), \quad (0:z_n)\neq D $
		
	\end{theorem}
	\begin{proof}
		Sea $ \{X_1,\cdots, X_n\} $ un conjunto finito de generadores de $ M $. Sea $ F $ un $ D $-módulo libre de rango $ n $ con base $ \{e_1, \cdots e_n\} $, y sea $ \varepsilon:F\flechaSobreyectiva M $ el homomorfismo sobreyectivo $ \varepsilon(e_i)=x_i, \text{ con } 1\leq i\leq n $.
		
		Sea $ N:=\text{Ker}(\varepsilon) $:
		\[ N\flechaInyectiva F \xFlechaSobreyectiva{\ \varepsilon \ } M \]
		Por la propiedad previa, $ N $ es libre de rango $ \leq n $ y en particular, $ N $ es de tipo finito. Sea $ \{f_1, \cdots, f_n\} $ un conjunto finito de generadores de $ N $ (como caso particular, puede ser una base, pero en la demostración no se necesita que lo sea, ni tampoco se necesita suponer que $ m\leq n $). Entonces:
		\[ f_i = \sum_{j=1}^{n}a_{ij}e_j, \quad 1\leq i\leq m, a_{ij}\in D \]
		dado que $ \sum_{i=1}^{n} a_{ij}x_j=0 \ \forall i$  (ya que $ \varepsilon(f_i)=\varepsilon(\sum_{i=1}^{n} a_{ij} e_j)\Rightarrow \varepsilon(f_i)=  \sum_{i=1}^{n} a_{ij}x_j \in \text{Ker}(\varepsilon)\Rightarrow \sum_{i=1}^{n} a_{ij}x_j=0$).
		
		Sea $ A:=(a_{ij})\in M_{m \times n}(D) $. Se tiene :
		\[ f=Ae \]
		es decir:
		
		\[ \begin{pmatrix}
			 f_1 \\
			 f_2 \\
			 \vdots \\
			 f_m
		\end{pmatrix} = \begin{pmatrix}
		a_{11} & a_{12} & \cdots & a_{1n} \\
		a_{21} & a_{22} & \cdots & a_{2n} \\
		\vdots & \vdots & \ddots & \vdots \\
		a_{m1} & a_{m2} & \cdots & a_{mn} \\
		\end{pmatrix} \cdot \begin{pmatrix}
		e_1 \\
		e_2 \\
		\vdots \\
		e_m
		\end{pmatrix}\]
		lo cual no es un producto de matrices en el sentido usual, ya que $ f,e $ no son matrices sobre $ D $, pero la notación será igualmente útil. Se comprueba fácilmente la asociatividad usual:
		\[ B(cf)=(Bc)f \]
		para matrices ordinarias $ B $, $ C $ sobre $ D $.
		
		Por el Teorema de Diagonalización, la matriz $ A $ es equivalente a una matriz diagonal. $ B= \text{diag}(d_1, \cdots,\\ d_r, 0, \cdots, 0)$. 
		$ B= QAP,\  Q\in M_{m\times m} (D), P\in M_{n\times n} (D) $ matrices inversibles, $ d_i\neq 0 $, $ 1 \leq i \leq r$, $d_i \mid d_{i+1}, \text{ y } 1 \leq i \leq r-1 $.
		
		Sea $ f':=Qf, e'=P^{-1}e $.
		Por el lema previo, $ f' $ es un conjunto de generadores de $ N $ y $ e' $ es una base de $ F $.
		Se tiene:
		\[ f'=Qf=QAe=QAPe'=Be' \]
		es decir:
		\[ f'_i=d_ie'_i, \quad 1 \leq i \leq r \]
		\[ f'_i =0,  \quad \text{si } r+1 \leq i \leq n\]
		([Importante en aplicaciones, no en la demostración] Nótese que como los $ e'_i $ son linealmente independientes y los $ d_i $ son no nulos, resulta que $ f'_1, \cdots, f'_r $ son linealmente independientes. Por tanto, $ N $ es libre de rango $ r $ y $ \{ f'_1, \cdots, f'_r\} $ es base de $ N $. Esto proporciona un método para encontrar una base de un submódulo de un $ D $-módulo libre a partir de un conjunto de generadores).
		
		Sea $y_i := \varepsilon(e'_i), 1 \leq i \leq n$. (Nótese que $ y=P^{-1}X $, ya que aplicando $\varepsilon $, con $ e'=P^{-1}e $, tenemos $ y=\varepsilon(e')=P^{-1}\varepsilon(e)=P^{-1}x $).\\
		Como $ \varepsilon $ es sobre, resulta que $ \{y_1, \cdots y_n \} $ es un conjunto de generadores de $ M $. 
		\[ M=\sum_{i=1}^{n}Dy_i\]
		Veamos que esta suma es directa ($ \Leftrightarrow \sum_{i\in I} x_i=0, x_i\in M_i, \forall i \Rightarrow x_i=0\ \forall i $), i.e.,
		\[ \sum_{j=1}^{n} b_i\cdot y_i=0, \quad b_i \in D \Rightarrow b_i\cdot y_i=0, \forall i \]
		Se tiene: 
		\[ 0=\sum_{i=1}^{n} b_i \cdot y_i = \sum_{i=1}^{n} b_i \cdot \varepsilon(e'_i)=\varepsilon (\sum_{i=1}^{n} b_i \cdot e'_i)\Rightarrow \sum_{i=1}^{n} b_i \cdot e'_i \in \ker(\varepsilon)=N \]
		\[ \Rightarrow \sum_{i=1}^{n} b_i \cdot e'_i=\sum_{i=1}^{r} c_i \cdot f'_i= \sum_{i=1}^{r} c_i \cdot d_i \cdot e'_i \Rightarrow (e'\text{ base}) \begin{cases}
		b_i=0, \quad r+1 \leq i \leq n \Rightarrow b_i \cdot y_i =0\\
		b_i=c_i\cdot d_i, \quad 1 \leq i \leq r \Rightarrow b_i\cdot y_i=0
		\end{cases}\]
		Ya que $ b_i\cdot y_i=c_id_iy_i=c_id_i\varepsilon(c_id_ie'_i)=\varepsilon(c_if'_i)=0 $($c_if_i\in \ker(\varepsilon)=N$).
		
		Así pues, 
		\[ M=Dy_1 \oplus \cdots \oplus Dy_n\]
		Veamos que:
		\[ (0:y_i)=(d_i), \quad  1 \leq i \leq r\]
		\[ (0:y_i)=0, \quad  r+1 \leq i \leq n\]
		"$ \supset $"\ ay está demostrado (trivial).\\
		"$ \subset $"\  :\[ \lambda \in (0:y_i) \Rightarrow 0=\lambda y_i =\lambda \varepsilon(e'_i)= \varepsilon(\lambda e'_i) \Rightarrow \lambda e'_i \in \ker(\varepsilon)=N\Rightarrow \] 
		\[\Rightarrow\lambda e'_i = \sum_{j=1}^{r}\mu_jf_j \Rightarrow \lambda e'_i = \sum_{j=1}^{r} \mu_j d_j e'_j \Rightarrow (e' \text{ base})\begin{cases}
		\lambda_i=0, \quad  r+1 \leq i \leq n \\
		\lambda_i=\mu_i, \quad 1 \leq i \leq r 
		\end{cases} \]
		Ahora, las relaciones $ d_1 \mid d_2 \mid \cdots \mid d_r $, implican $ (d_1)\supset (d_2) \supset \cdots \supset (d_r) $, es decir, $ (0:y_1)\supset(0:y_2)\supset\cdots\supset(0:y_r) $, y por tanto: 
		\[ (0:y_1)\supset(0:y_2)\supset\cdots\supset(0:y_r)\supset(0:y_{r+1})\supset\cdots\supset(0:y_n)\]
		
		Sea $ s $, $ 0 \leq s \leq n $, tal que $ (0:y_1)=(0:y_2)=\cdots =(0:y_{n-s})=D $, $ (0:y_{n-s+1})\neq D $. \\Entonces: $ y_1=y_2=\cdots = y_{n-s}=0 $.
		
		A continuación, definimos: 
		\[ z_1:=y_{n-s+1},z_2:=y_{n-s+2}, \cdots, z_s:=y_n  \]
		Se tiene: 
		\[ M=Dz_1\oplus \cdots \oplus Dz_s \]
		y $ (0:z_1)\supset(0:z_2)\supset\cdots \supset(0:z_s)$, con $ (0:z_k)\neq D $.
	\end{proof}
	\begin{note}
		Sea $ M $ un $ D $-módulo con generadores $ x_1,\cdots, x_n $ y relaciones $\sum_{j=1}^{n}a_{ij}e_j=0, 1 \leq i \leq m$. La descomposición de $ M $ dada por el Teorema de Estructura se calcula, en resumen, del siguiente modo:
		\[ A:=(a_{ij}) \in M_{m \times n}(D) \]
		Se diagonaliza: $ QAP=B= \text{diag}(d_1, \cdots d_r, 0, \cdots, 0),  d_1 \mid d_2 \mid \cdots \mid d_r$. Se tiene que $ Y=P^{-1}X $. Entonces:
		\[ M=Dy_1\oplus \cdots \oplus Dy_n \]
		\[ (0:y_i)=(d_i), \quad  1 \leq i \leq r\]
		\[ (0:y_i)=0, \quad  r+1 \leq i \leq n\]
		Por último, eliminamos los $y_i$ que sean nulos y de esta forma se obtienen los $ z_k $ del enunciado del Teorema.
		 
	\end{note}
	%%jorge 29-11
	
	\begin{corollary} %%ESTO ES EN REALIDAD UN COROLARIO
		Si $D$ es un DIP y $M$ es un módulo de tipo finito (t.f.), entonces $M$ es 
		suma directa de un submódulo de torsión y de un submódulo libre.
	\end{corollary}
	\begin{proof}
		Por el teorema de estructuras tenemos que: 
		\[M = DZ_1 \oplus \cdots \oplus DZ_s\]
		\[(0:Z_1) \supset \cdots \supset (0:Z_s), (0:Z_k) \neq 0.\]
		Sea $r$, $0 \leq r \leq s$, tal que $(0:Z_i) \neq 0$ para $0 \leq i \leq r$ y
		$(0:Z_i) = 0$ para $r \leq i \leq s$. $\\$
		Entonces:
		\[DZ_1 \oplus \cdots \oplus DZ_r = T(M)\]
		y $DZ_1 \oplus \cdots \oplus DZ_r$ es libre.
		\[DZ_1 \oplus \cdots \oplus DZ_r \oplus DZ_{r+1} \oplus \cdots \oplus DZ_s\]
		"$\subset$" $\\$
		$1 \leq i \leq r$: $(0:Z_i) \neq 0$ $\Rightarrow$ $Z_i \in T(M)$ $\Rightarrow$
		$DZ_i \subset T(M)$ $\Rightarrow$ $DZ_1 \oplus \cdots \oplus DZ_r \subset T(M)$ $\\$
		
		"$\supset$" $\\$
		$x \in T(M)$ $\Rightarrow$ $x \in M$ $\Rightarrow$ $x = \sum_{i=1}^{s}\lambda_i z_i$, 
		$\lambda_i \in D$ $\\$
		Además, como $x \in T(M)$ tenemos que:
		\[\exists a \in D, a\neq 0 / ax = 0\]
		y que:
		\[a(\sum_{i=1}^{s}\lambda_i z_i) = \sum_{i=1}^{s} a \lambda_i z_i \in DZ_i\]
		con lo que podemos extraer que, por ser una suma directa:
		\[a \lambda_i z_i \in DZ_i \Rightarrow a\lambda_i \in (0:Z_i) \Rightarrow a_i\lambda_i = 0, \hspace{0.2cm} r + 1 \leq i \leq s \stackrel{a \neq 0}{\Rightarrow} \lambda_i = 0, \hspace{0.2cm} r + 1 \leq i \leq s\]
		Por lo que, 
		\[x = \sum_{i = 1}^{r} \lambda_i z_i \in DZ_1 \oplus \cdots \oplus DZ_r\]
		y, 
		\[DZ_{r+1} \oplus \cdots \oplus DZ_s \simeq D \oplus \stackrel{s - r}{\cdots} \oplus D = D^{s - r} \hspace{0.2cm} libre.\]
		Por lo que, para $r + 1 \leq i \leq s$, 
		\[DZ_i \simeq \frac{D}{(0:Z_i)} = D.\]
	\end{proof}
	\begin{corollary}
		Sea $D$ un DIP y $M$ un $D$-módulo de torsión de tipo finito. Entonces:
		\begin{center}
			$M$ es libre $\Leftrightarrow$ $M$ es libre de torsión.
		\end{center}
	\end{corollary}
	\begin{proof}
		Inmediata a partir del anterior corolario.
	\end{proof}
	
	\section{Módulos de torsión y componentes primarias}
	Sea $D$ un DIP y $M$ sea $D$-módulo de tipo finito. Sea además $p$ un elemento primo de $D$ (i.e., $(p)$ es un ideal primo distinto de $0$, i.e., $p$ es irreducible). $\\$
	Se llama \underline{componente $p$-primaria} de $M$ al subconjunto:
	\[M_p = \{y\in M / \exists k\in \mathbb{N}, p^{k}y = 0\},\]
	que, como se puede observar, son los elementos de $M$ aniquilados por alguna potencia de $p$.$\\$
	Además, $M_p$ es un submódulo de $M$ y $M_p \subset T(M)$. Así, si $M_p = M$, se dice que $M$ es \underline{$p$-primario}.
	\begin{example}
		Cada $D$-módulo cíclico de la forma $\frac{D}{T(M)}$ es $p$-primario. Tenemos que
		\[f: D \longrightarrow \frac{D}{(p^e)}\]
		es la aplicación canónica. Vemos que:
		\[p^e[\frac{D}{(p^e)}] = 0\]
		y
		\[p^e(x + (p^e)) = p^ex + (p^e) = 0.\]
	\end{example}
	\begin{lemma}
		Si $p_1, ..., p_h$ son primos distintos, entonces la suma de las componentes $p_i$-primarias, $1\leq i\leq h$ es suma directa:
		\[\sum_{i=1}^{h} = \oplus_{i=1}^h Mp_i.\]
	\end{lemma}
	\begin{proof}
		Probar que es suma directa probaremos lo siguiente:
		\[M_{p_1} \cap [M_{p_2} + \cdots M_{p_h}] \stackrel{?}{=} 0.\]
		Supondremos $y \in M_{p_1} \cap [M_{p_2} + \cdots M_{p_h}]$, entonces:
		\[y \in M_{p_1} \Rightarrow p_1^{k_1}y=0\]
		e:
		\[y \in M_{p_2} + \cdots M_{p_h} \Rightarrow p_2^{k_2} \cdots p_h^{k_h}y = 0,\]
		pero mcd$(p_1^{k_1}, p_2^{k_2}\cdots p_h^{k_h}) = 1$ y, por lo tanto:
		\[1 \in (0:y) \Rightarrow y=0.\]
	\end{proof}
	\begin{lemma}
		Sea $D$ un DIP y $M$ un $D$-módulo. Sea $x\in M$ y $d\in D$ tal que $(0:x) = (d)$. Sea $d = np_1^{e_1} \cdots p_t^{e_t}$, $p_i \neq p_j$, si $i\neq j$, la factorización de $d$ en primos. Entonces:
		\[DX = DX_1 \oplus \cdots \oplus DX_t,\]
		con $(0:X_i) = (p_i^{e_i})$. En particular, $DX_i$ es $p$-primario. 
	\end{lemma}
	\begin{proof}
		Para cada $i$, $1 \leq i \leq t$, sea
		\[d_i = n p_1^{e_1} \cdots p_{i-1}^{e_{i+1}} \cdots p_t^{k_t}\]
		de modo que $d = d_ip_i^{e_i}$. Además, sea $X_i = d_iX \in M$. Se tiene entonces que $(0:X_i) = (p_i^{e_i})$. Veámoslo:$\\$
		"$\supset$"
		\[p_i^{e_i}X_i = p_i^{e_i} d_i X = dX = 0.\]
		"$\subset$"
		\[x\in (0:X_i) \Rightarrow 0 = \lambda X_i = \lambda d_i X \Rightarrow \lambda d_i \in (0:X) = (d) \Rightarrow \lambda d_i = \mu d \Rightarrow \lambda d_i= \mu d_i p_i^{e_i} \Rightarrow \lambda = \mu p_i^{e_i} \in (p_i^{e_i}).\]
		Veamos que $DX = DX_1 \oplus \cdots \oplus DX_t$:
		\begin{center}
			$(0: X_i) = (p_i^{e_i}) \Rightarrow x_i \in M_{p_i} \Rightarrow DX_i \subset M_{p_i} \stackrel{Lema 3.1}{\Rightarrow}$ La suma $DX_1 + \cdots + DX_t$ es directa.
		\end{center}
		"$\supset$"$\\$
		\[X_i = d_i X\in DX\]
		"$\subset$"$\\$
		\[mcd(d_1, ..., d_t)\Rightarrow 1 = \sum_{i = 1}^{t} \alpha_i d_i,\]
		con $\alpha_i \in D$. Entonces:
		\[X = 1X = (\sum_{i = 1}^{t} \alpha_i d_i)X = \sum_{i = 1}^{t} \alpha_i (d_iX) = \sum_{i = 1}^{t} \alpha_i X_i \Rightarrow X \in DX_1 + ... + DX_t \Rightarrow DX \subset DX_1 + ... + DX_t.\]
	\end{proof}
	\begin{proposition}
		Sea $D$ un DIP y $M$ un $D$-módulo de torsión de tipo finito. Entonces:
		\[M = \oplus_p M_p,\]
		donde la suma se extiende a todos los primos $p$ tales que $M_p \neq 0$, y existe solamente un número finito de tales primos. $\\$
		Cada $M_p$ se puede expresar de la forma
		\[M_p = DZ_1 \oplus \cdots \oplus DZ_r\]
		tal que $(0:Z_i) = (p^{e_i})$ y $1 \leq e_1 \leq ... \leq e_r$.
		Además la sucesión $e_1, ..., e_r$ está determinada de modo único.
	\end{proposition}
\end{document}





