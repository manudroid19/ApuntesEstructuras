

%-----------------------------------------------------------------------------------------------------  
%	INCLUSIÓN DE PAQUETES BÁSICOS.
%-----------------------------------------------------------------------------------------------------

\documentclass{article}
\usepackage{enumerate}
\usepackage{wrapfig}
\usepackage{graphicx}



%-----------------------------------------------------------------------------------------------------
%	SELECCIÓN DEL LENGUAJE
%-----------------------------------------------------------------------------------------------------

% Paquetes para adaptar Látex al Español:
\usepackage[spanish,es-noquoting, es-tabla, es-lcroman]{babel} % Cambia
\usepackage[utf8]{inputenc}                                    % Permite los acentos.
\selectlanguage{spanish}                                       % Selecciono como lenguaje el Español.

%flechassssss
\def\flechaSobreyectiva{\mathrel{\mkern16mu  \vcenter{\hbox{$\scriptscriptstyle+$}}%
		\mkern-25mu{\longrightarrow}}}
\def\flechaInyectiva{\mathrel{\mkern0mu  \vcenter{\hbox{$\scriptscriptstyle+$}}%
		\mkern-9mu{\longrightarrow}}}
\def\flechaBiyectiva{\mathrel{\mkern16mu  \vcenter{\hbox{$\scriptscriptstyle+$}}\mkern-25mu{\mathrel{\mkern0mu\vcenter{\hbox{$\scriptscriptstyle+$}}\mkern-9mu{\longrightarrow}}}}}
\def\xFlechaInyectiva #1{\mathrel{\ooalign{\thinspace\thinspace$\mapstochar\mkern5mu$\hfil\cr$\xrightarrow{#1}$\cr}}}
\def\xFlechaSobreyectiva #1{\mathrel{\ooalign{\hfil$\mapstochar\mkern5mu$\thinspace\thinspace\cr$\xrightarrow{#1}$\cr}}}
%llave a la derecha
\newenvironment{rcases}
{\left.\begin{aligned}}
	{\end{aligned}\right\rbrace}

%-----------------------------------------------------------------------------------------------------
%	SELECCIÓN DE LA FUENTE
%-----------------------------------------------------------------------------------------------------

% Fuente utilizada.
\usepackage{courier}                    % Fuente Courier.
\usepackage{microtype}                  % Mejora la letra final de cara al lector.

%-----------------------------------------------------------------------------------------------------
%	ESTILO DE PÁGINA
%-----------------------------------------------------------------------------------------------------

% Paquetes para el diseño de página:
\usepackage{fancyhdr}               % Utilizado para hacer títulos propios.
\usepackage{lastpage}               % Referencia a la última página. Utilizado para el pie de página.
\usepackage{extramarks}             % Marcas extras. Utilizado en pie de página y cabecera.
\usepackage[parfill]{parskip}       % Crea una nueva línea entre párrafos.
\usepackage{geometry}               % Asigna la "geometría" de las páginas.

% Se elige el estilo fancy y márgenes de 3 centímetros.
\pagestyle{fancy}
\geometry{left=3cm,right=3cm,top=3cm,bottom=3cm,headheight=1cm,headsep=0.5cm} % Márgenes y cabecera.
% Se limpia la cabecera y el pie de página para poder rehacerlos luego.
\fancyhf{}

% Espacios en el documento:
\linespread{1.1}                        % Espacio entre líneas.
\setlength\parindent{0pt}               % Selecciona la indentación para cada inicio de párrafo.

% Cabecera del documento. Se ajusta la línea de la cabecera.
\renewcommand\headrule{
	\begin{minipage}{1\textwidth}
		\hrule width \hsize
	\end{minipage}
}

% Texto de la cabecera:
\lhead{\docauthor}                          % Parte izquierda.
\chead{}                                    % Centro.
\rhead{\subject \ - \doctitle}              % Parte derecha.

% Pie de página del documento. Se ajusta la línea del pie de página.
\renewcommand\footrule{
	\begin{minipage}{1\textwidth}
		\hrule width \hsize
	\end{minipage}\par
}

\lfoot{}                                                 % Parte izquierda.
\cfoot{}                                                 % Centro.
\rfoot{Página\ \thepage\ de\ \protect\pageref{LastPage}} % Parte derecha.

%----------------------------------------------------------------------------------------
%	MATEMÁTICAS
%----------------------------------------------------------------------------------------

% Paquetes para matemáticas:
\usepackage{amsmath, amsthm, amssymb, amsfonts, amscd} % Teoremas, fuentes y símbolos.

% Nuevo estilo para definiciones
\newtheoremstyle{definition-style} % Nombre del estilo
{5pt}                % Espacio por encima
{5pt}                % Espacio por debajo
{}                   % Fuente del cuerpo
{}                   % Identación: vacío= sin identación, \parindent = identación del parráfo
{\bf}                % Fuente para la cabecera
{.}                  % Puntuación tras la cabecera
{.5em}               % Espacio tras la cabecera: { } = espacio usal entre palabras, \newline = nueva línea
{}                   % Especificación de la cabecera (si se deja vaía implica 'normal')

% Nuevo estilo para teoremas
\newtheoremstyle{theorem-style} % Nombre del estilo
{5pt}                % Espacio por encima
{5pt}                % Espacio por debajo
{\itshape}           % Fuente del cuerpo
{}                   % Identación: vacío= sin identación, \parindent = identación del parráfo
{\bf}                % Fuente para la cabecera
{.}                  % Puntuación tras la cabecera
{.5em}               % Espacio tras la cabecera: { } = espacio usal entre palabras, \newline = nueva línea
{}                   % Especificación de la cabecera (si se deja vaía implica 'normal')

% Nuevo estilo para ejemplos y ejercicios
\newtheoremstyle{example-style} % Nombre del estilo
{5pt}                % Espacio por encima
{5pt}                % Espacio por debajo
{}                   % Fuente del cuerpo
{}                   % Identación: vacío= sin identación, \parindent = identación del parráfo
{\scshape}                % Fuente para la cabecera
{:}                  % Puntuación tras la cabecera
{.5em}               % Espacio tras la cabecera: { } = espacio usal entre palabras, \newline = nueva línea
{}                   % Especificación de la cabecera (si se deja vaía implica 'normal')

% Teoremas:
\theoremstyle{theorem-style}  % Otras posibilidades: plain (por defecto), definition, remark
\newtheorem{theorem}{Teorema}[section]  % [section] indica que el contador se reinicia cada sección
\newtheorem{corollary}[theorem]{Corolario} % [theorem] indica que comparte el contador con theorem
\newtheorem{lemma}[theorem]{Lema}
\newtheorem{proposition}[theorem]{Proposición}

% Definiciones, notas, conjeturas
\theoremstyle{definition}
\newtheorem{definition}{Definición}[section]
\newtheorem{conjecture}{Conjetura}[section]
\newtheorem*{note}{Nota} % * indica que no tiene contador
\newtheorem*{observation}{Observación} % * indica que no tiene contador
\newtheorem*{properties}{Propiedades}
\newtheorem*{comment}{Comentario clase}

% Ejemplos, ejercicios
\theoremstyle{example-style}
\newtheorem{example}{Ejemplo}[section]
\newtheorem{exercise}{Ejercicio}[section]

%-----------------------------------------------------------------------------------------------------
%	PORTADA
%-----------------------------------------------------------------------------------------------------

% Elija uno de los siguientes formatos.
% No olvide incluir los archivos .sty asociados en el directorio del documento.
%\usepackage{title1}
\usepackage{title2}
%\usepackage{title3}

%-----------------------------------------------------------------------------------------------------
%	TÍTULO, AUTOR Y OTROS DATOS DEL DOCUMENTO
%-----------------------------------------------------------------------------------------------------

% Título del documento.
\newcommand{\doctitle}{Tema 3. Módulos}
% Subtítulo.
\newcommand{\docsubtitle}{}
% Fecha.
\newcommand{\docdate}{5 \ de \ Noviembre \ de \ 2018}
% Asignatura.
\newcommand{\subject}{Estructuras algebraicas}
% Autor.
\newcommand{\docauthor}{Sergio Mayo, Manuel de Prada, Jorge Vázquez}
\newcommand{\docaddress}{Universidade de Santiago de Compostela}
\newcommand{\docemail}{lalala@gmail.com}

%-----------------------------------------------------------------------------------------------------
%	RESUMEN
%-------------------------------					----------------------------------------------------------------------

% Resumen del documento. Va en la portada.
% Puedes también dejarlo vacío, en cuyo caso no aparece en la portada.
%\newcommand{\docabstract}{}
\newcommand{\docabstract}{Tema de módulos del peazo profe Javier Majaderias.}

\begin{document}

\maketitle

%-----------------------------------------------------------------------------------------------------
%	ÍNDICE
%-----------------------------------------------------------------------------------------------------

% Profundidad del Índice:
%\setcounter{tocdepth}{1}

\newpage
\tableofcontents
\newpage

%----------------------------------------------------------------------------------------
%	Sección 1: Deficiones y teoremas
%----------------------------------------------------------------------------------------

\section{Módulos}
	Sea $A$ un anillo conmutativo.
    \begin{definition}
        Un $A$-módulo es un grupo abeliano $M$ con una operación:
        \[A\times M\longrightarrow M \]
        \[(a,m) \longmapsto a\cdot m =am\]
        que verifica:
        \begin{enumerate}[\hspace{1cm}i)]
        	\item $(a+b)m = am+bm$
        	\item $a(m+n)=am+an$
        	\item $(ab)m = a(bm)$
        	\item $1m=m$
        \end{enumerate}
    \end{definition}

    \begin{observation}
    	$A$ es un $A$-módulo
    	\[A\times A\longrightarrow A \]
    	\[(a,b) \longmapsto a\cdot b\]
    \end{observation}

    \begin{properties}
    	\
    	\begin{enumerate}
    		\item $0m=0$ ya que $0m=(0+0)m=0m+0m$
    		\item $a0=0$
    		\item $(-a)m=-(am)=a(-m)$
    	\end{enumerate}
    \end{properties}
\begin{observation}
	$am=0 \nRightarrow \begin{cases}
	a=0\\
	\quad\text{ó}\\
	m=0
	\end{cases}$
\end{observation}
\begin{observation}
	Como $A$ conmutativo, $ma:=am$
	\[m(ab)=(ab)m=a(bm)=mb(a)=m(ba)\]
\end{observation}
\begin{proposition}
	Sea $M$ grupo abeliano.
	Entonces $M$ es un $\mathbb{Z}$-módulo.
	\\Nota clase: la estructura de un $\mathbb{Z}$ módulo y de grupo abeliano es la misma
\end{proposition}
\begin{proof}
	Sea $n \in \mathbb{Z}, x\in M$
	\[n\cdot x= \begin{cases}
		x+...+x&\quad \text{si }n>0\\
		0 &\quad \text{si } n=0\\
		(-x)+...+(-x)=-(-n)x &\quad \text{si }n<0
	\end{cases}
	 \]
	$nx=(1+...+1)x=1x+...1x=x+...+x$
\end{proof}
\begin{note}
	Descomposición en primos:\\
	$x^2-y^2=n \\
	(x+y)(x-y)=n=p_1 p_2 p_3 \\
	(x+iy)(x-iy)=n $ (no tiene sentido hablar de primos en los números complejos(?))
\end{note}

\begin{example}
	Utilización anillos: cálculo raíces polinomios.\\
	$f(x,y)=y^2-x^3-x^2 $ \\
	$A=\frac{K[x,y]}{f(x,y)}$ aporta información sobre las raíces.
\end{example}

\begin{comment}
	¿Cual es la motivación de estudiar los módulos? \\
	Aparecen en representaciones de grupos:	Sabiendo los módulos de un anillo, se puede saber como es un anillo.\\
	Otra razón (se ve en el máster): Si un anillo es regular, es útil en el estudio de puntos no singulares.\\
	Cada punto de la curva esta determinado por un anillo. Si algunos anillos son regulares, otros también lo son. Es un vago ejemplo de que los módulos tienen utilidad.
\end{comment}

\begin{definition}
	Sea $M$ un $A$-módulo, $N$ subconjunto de $M$.\\
	Se dice que $N$ es un $A$-submódulo de $M$ si $N$ es un $A$-módulo con la operación de $A$ en $M$.
	\begin{observation}
		\
		\begin{enumerate}
			\item $ \forall y \in N, a \in A \quad\Rightarrow ay\in N$
			\item $\forall x,y\in N  \quad \Rightarrow x-y \in N$
			\item $\{0\}, M$ son submódulos de $M$
			\item Si $I$ es un ideal de $A$, $M$ un $A$-módulo, entonces
			\[IM:=\{ \sum_{i=1}^{n} a_i\cdot x_i \quad : n \in N, a_i \in I, x_i \in M\}\]
			es un submódulo de M. ($a \cdot \sum a_i x_i = \sum (a\cdot a_i) \cdot x_i$, nótese que $aa_i\in I$)

			\item Si $I$ es un subconjunto de $A$, entonces $ I $ es $ A $-submódulo de $ A \Leftrightarrow I $ es ideal de $ A $ ($ A $ como $ A $-módulo).
		\end{enumerate}
	\end{observation}
		\begin{definition}
			Si $ M_1 $ y $ M_2 $ son submódulos de un $ A $-módulo M se define la suma de $M_1 $ y $ M_2$:
			\[ M_1+M_2 = \{x_1+x_2 \in M : x_1\in M_1, x_2 \in M_2\}\]
			Es el menor submódulo de $ M $ que contiene a $ M_1 $ y $ M_2 $.\\
			Nótese:
			\begin{enumerate}
				\item $ x_1+x_2-(x_1'+x_2') = (x_1-x_1')+ (x_2-x_2') \in M_1+M_2 $
				\item $ a(x_1+x_2)=a(x_1)+a(x_2)\in M_1+M_2 $
			\end{enumerate}
		\end{definition}

	\begin{definition}
		Sea $ \{M_i\}_{i\in I} $ familia de submódulos de $ M $. Definimos en general
		\begin{align*}
			\sum_{i\in I}M_i &= \{\sum_{i\in I}x_i \in M : x_i\in M_i \text{para cada i, }a_i\in A, \text{casi todos los }x_i=0\}\\
			&= \{\text{sumas finitas de elementos de }\bigcap M_i\}=x_1+...+x_r
		\end{align*}

		Es el menor submódulo de $ M $ que contiene $ \bigcup_{i\in I} M_i $
	\end{definition}

\end{definition}

JORGE 6-11

\begin{definition}
	Sea $A$ un anillo. Un $A$-módulo $M$ es cíclico si está generado por un elemento, es decir:
	\[\exists x  \in M \ / \   M=<x>=Ax\]
\end{definition}

\begin{observation}
	Sea $f:A\rightarrow B$ un homomorfismo de anillos. Entonces $B$ es un $A$-módulo definiendo la operación
	externa como:\[
		a\in A,b\in B \ \ \ \ a\cdot b :=f(a)b
	\]
	En particular si $I$ es un ideal de $A$, $A/I$ es un $A$-módulo ($p:A\rightarrow A/I$).
\end{observation}

\begin{proposition}
	Si $M$ es un $A$-módulo son equivalentes:
	\begin{enumerate}[\hspace{1cm}i)]
		\item $M$ es cíclico.
		\item Existe un homomorfismo sobreyectivo de $A$-módulos $\phi :A\rightarrow M$
		\item Existe un ideal $I$ de $A$ y un isomorfismo de $A$-módulos $A/I \simeq M$
	\end{enumerate}
\end{proposition}

\begin{proof} \ \\
	$i)\Rightarrow ii)$:
	\[ M \text{ cíclico}\Rightarrow M=<x>\]
	\[ \phi :a \in A\rightarrow \phi (a):=ax \in M \text{ es un homomorfismo de $A$-módulos sobreyectivo:}\]
	\[ \phi(a+b)=(a+b)x=ax+bx=\phi(a)+\phi(b)\]
	\[ \phi(\lambda a)=(\lambda a)x= \lambda(ax)=\lambda\phi(a) \ \ \ \forall\lambda\in A \]
	\[ Im(\phi)=\{ax \ / \ a\in A \} = <x>=M\]
	$ii)\Rightarrow iii)$:
	\[ \text{Por hipótesis existe } \phi:A\rightarrow M \text{ homomorfismo sobreyectivo de $A$-módulos.}\]
	\[ A/Ker(\phi) \simeq M \text{ y $Ker(\phi)$ es un ideal de } A \]
	$iii)\Rightarrow i)$:
	\[ A/I \text{ es un $A$-módulo cíclico.} \]
	\[ A/I=<\bar{1}>\ \Rightarrow\text{$M$ es cíclico} \]
\end{proof}

\begin{definition}
	Sean $A$ anillo y $M$ $A$-módulo. Se llama aniquilador de $M$ al ideal de $A$ definido por:
	\[(0:M):=\{a\in A \ / \ ax=0 \  \forall x \in M\}\]
\end{definition}

\begin{definition}
	Tomando $M=<x>$ definimos:
	\[(0:x):=(0:<X>)=\{a\in A \ / \ ax=0\}\]
\end{definition}

\begin{proposition}
	Sea $A$ un anillo y $M$ un $A$-módulo. Se verifica que en efecto $(0:M)$ es un ideal.
\end{proposition}

\begin{proof}
	\[ \text{Trivialmente } 0 \in (0:M) \Rightarrow (0:M) \neq \text{\O}\]
	\[ a,b \in (0:M) \Rightarrow ax=0=bx \ \forall x \in M \Rightarrow (a-b)x=ax-bx=0-0=0 \ \forall x \in M \Rightarrow a-b \in (0:M) \]
	\[ \lambda \in A, a \in (0:M) \Rightarrow ax=0 \ \forall x \in M \Rightarrow (\lambda a)x=0 \ \forall x \in M \Rightarrow \lambda a \in (0:M) \]
\end{proof}

\begin{observation}
	Sea $x \in M$, definimos:
	\[ A\xrightarrow{\ \ \phi \ \ } M \text{ homomorfismo de $A$-módulos}\]
	\[ \phi(a):=ax\]
	\[ Ker(\phi)=(0:x)\]
	\[A/(0:x)\xrightarrow{\ \ \cdot x\ \ }M \text{ homomorfismo inyectivo}\]
\end{observation}

\begin{observation}
	Tomando $A=\mathbb{Z}$:
	\[ \mathbb{Z}\text{-módulo}=\text{ grupo abeliano}\]
	\[ \mathbb{Z}\text{-módulo cíclico}=\text{ grupo cíclico}\]
\end{observation}

\begin{definition}
	Sea $\{M_i\}_{i\in I}$ una familia de $A$-módulos. Definimos el producto directo de los $M_i$ como el $A$-módulo:
	\[ \prod_{i\in I}M_i := \{f:I\rightarrow\bigcup_{i\in I}M_i \ / \ \text{$f$ aplicación}, f(i)\in M_i \ \forall i \in I\} = \{(x_i)_{i\in I} \ / \ x_i \in M_i \ \forall i \in I\}\]
	\[(x_i)_{i\in I} + (y_i)_{i\in I} := (x_i + y_i)_{i\in I}\]
	\[a(x_i)_{i\in I} := (ax_i)_{i\in I}\]
\end{definition}

\begin{observation}
	En el caso finito, los elementos del producto los representamos con n-tuplas:
	\[(x_1,\dots,x_n)+(y_1,\dots,y_n)=(x_1+y_1,\dots,x_n+y_n)\]
	\[a(x_1,\dots,x_n)=(ax_1,\dots,ax_n)\]
\end{observation}

\begin{observation}
	La proyección habitual $\pi_j:\prod_{i\in I}M_i \longrightarrow M_j$ es un homomorfismo de $A$-módulos para todo $j$.
\end{observation}

\begin{definition}
	Sea $\{M_i\}_{i\in I}$ una familia de $A$-módulos. Definimos la suma directa de los $M_i$ como el $A$-submódulo de $\prod_{i\in I}M_i$:
	\[\bigoplus_{i\in I}M_i=\{(x_i)_{i\in I} \in \prod_{i\in I}M_i \ / \ x_i=0 \text{ para casi todo } i \in I\}\]
\end{definition}

\begin{observation}
	La aplicación
	\[M_j:M_j\longrightarrow\bigoplus_{i\in I}M_i\]
	\[M_j(m):= (x_i)_{i\in I} \ / \ x_j=m,\  x_i=0 \ \forall i\neq j \]
	es un homomorfismo de $A$-módulos inyectivo para cualquier $j$.
\end{observation}

\begin{observation}
	Si $I$ es finito:
	\[\prod_{i\in I}M_i = \bigoplus_{i\in I}M_i \]
\end{observation}

\begin{proposition}
	Sea $M$ un $A$-módulo y $\{M_i\}_{i\in I}$ una familia de $A$-submódulos de $M$. La aplicación:
	\[ \bigoplus_{i\in I}M_i \longrightarrow \sum_{i\in I}M_i \]
	\[ (x_i)_{i\in I} \leadsto \sum_{i\in I}x_i \]
	es un homomorfismo de $A$-módulos.
\end{proposition}

\begin{proposition}
	Sea $ \{M_i\}_{i\in I} $ familia de submódulos de $ M $. Consideremos:
	\[\phi : \oplus_{i\in I} M_i\longrightarrow \oplus_{i\in I }M_i \]
	\[\phi ((x_i)_{i\in I})=\sum_{i\in I} x_i \]
	Equivalen:
	\begin{enumerate}[\hspace{1cm}i)]
		\item $\phi$ isomorfismo
		\item $\sum_{i\in I}x_i=0,x_i \in M_i \ \forall i, x_i=0$ para casi todo i $\Rightarrow$ $ x_i=0 \ \forall i \in I $
		\item $M_j \cap (\sum_{i\neq j}M_i)\forall j \in I$
	\end{enumerate}
\end{proposition}

\begin{proof} \ \\
	$i) \Rightarrow ii)$:\\
	 \[ 0=\sum_{i\in I} x_i=\phi ((x_i)_{i\in I}) \Rightarrow \text{(ya que $\phi$ inyectiva) } (x_i) \in I =0 \Rightarrow x_i =0 \ \forall i \in I \]
	$ii) \Rightarrow iii)$:\\
	 \[ x \in M_j \cap (\sum_{i\neq j}M_i) \Rightarrow x=\sum_{i\in I, i\neq j, x_i\in M_i} x_i \Rightarrow\]
	 \[ \sum_{i\in I, i\neq j} x_i -x = 0 \  (\text{ya que }x\in M_j) \Rightarrow  x=0  \]

	$iii) \Rightarrow i)$:

	\parshape 1 2cm \dimexpr\linewidth-2cm\relax
	Sea $ (x_i)_{i\in I} \in \oplus_{i\in I} M_i  $ tal que $ \phi ((X_i)_{i\in I})=\sum_{i\in I} x_i=0 $. \\
	Tenemos que demostrar que $ x_i=0 \ \forall i \in I $. \\
	Sea $ j\in I, x_j = -\sum_{i\in I, i\neq j}x_i \in \sum_{i\in I} M_j \Rightarrow x_j \in M_j \cap (\sum_{i\neq j} M_i)=0 \Rightarrow x_j=0 $
\end{proof}
\begin{definition}
	$ A $ dominio. Sea $ M$ un $ A $-módulo. Un elemento $ x\in M $ es de torsión si $ (0:x)\neq 0 $.
	(si existe $ a\in A, a \neq 0 $ tal que $ ax=0 $).
	El conjunto de elementos de tensión de $ M $ se denota $ T(M) $ y es un $ A $-submódulo de $ M $.
\end{definition}
\begin{observation}
	\[x,y\in T(M) \Rightarrow \exists a,b \in A, a\neq 0\neq b \text{ con } ax=0, by=0\]
	\[ ab(x-y)=abx-aby=0-0=0\]
	\[0\neq ab \in A \Rightarrow x-y \in T(M)\]
\end{observation}
\begin{observation}
\[ a \in A, x \in T(M) \Rightarrow \exists a\neq b \in A : bx=0 \Rightarrow b(ax)=0 \Rightarrow ax \in T(M) \]
\end{observation}
\begin{definition}
	Si $ T(M)=M $ se dice que $ M $ es un $ A $-módulo de torsión. Si $ T(M)=0 $ se dice que $ M $ es un $ A $-modulo libre de torsión.
\end{definition}
\begin{proposition}
	\[  T(M) \hookrightarrow M \flechaSobreyectiva M/T(M)  \]

	$M/T(M)$ es libre de torsión.
\end{proposition}
\begin{proof}
	\ \\

	$\begin{rcases}
		\begin{rcases}
		\overline{x} = x + T(M) \in M/T(M) \\
		0\neq a \in A \quad a \cdot \overline{x} = \overline{a}
		\end{rcases}
	 \overline{ax}=\overline{0}  \Rightarrow ax\in T(M) \\
	 0\neq b b(ax)=0, (ba)x=0, ba\neq 0
	 \end{rcases}
	 x \in T(M) \Rightarrow  \overline{x} \neq \overline{0}
	 $
\end{proof}
\begin{exercise}
	$f: M \longrightarrow N$ homomorfismo de $ A $-módulo. Probar que $ f(T(M)) \subset T(N) $
\end{exercise}
\begin{definition}
	Sea $ M $ un $ A $-módulo.\\
	Una familia $ \{x_i\}_{i\in I} $ de elementos de $ M $ es linealmente independiente si:
	\[0=\sum_{i\in I}a_i x^i, a_i \in A \ \forall i, a_i =0 \text{ para casi todo }i \Rightarrow a_i =0 \ \forall i\in I \]
\end{definition}
\begin{observation}
	$ \{x_i\}_{i\in I} $ linealmente independiente $ \Leftrightarrow $ toda subfamilia finita de $ \{x_i\}_{i\in I} $ es linealmente independiente.
\end{observation}
\begin{observation}
	$ \{x_i\}_{i\in I} $ linealmente independiente.
		$ Ax_i = <x_i>$
		\[ \oplus_{i\in I}Ax_i \longrightarrow \phi \sum_{i\in I} A x_i \]
		\[ (a_i x_i)_{i\in I} \longmapsto \sum_{i\in I}a_i x_i \]  es sobreyectiva e inyectiva ya que $\sum_{i\in I}a_i x_i=0\Rightarrow a_i=0 \ \forall i\in I\Rightarrow (a_i x_i)_{i\in I} = (0)_{i\in I}$.
\end{observation}
\begin{example}
	\
	\begin{itemize}
		\item $ \{0\} $ no es un conjunto linealmente independiente $ a0=0, a\neq 0$.
		\item $ \{0\} \cup S$ no es linealmente independiente.	$ a0+\sum_{s\in S}0\cdot s =0 $.
	\end{itemize}
\end{example}
\begin{definition}
	$ \{x_i\}_{i\in I} $ linealmente independiente.
	\[ \psi:A\longrightarrow Ax_i=<x_i> \]
	\[ a\longmapsto \psi(a)=ax_i \]
	homomorfismo de $A$-módulo sobreyectivo. Además,
	\[ \text{Ker}(\psi)=(=:x_i)=\{a\in A : ax_i=0\} \]
	\[ A=A/(0:x_i)\simeq Ax_i \]
\end{definition}
	\begin{observation}
		$ \{x\} $ linealmente independiente $ \Leftrightarrow (0:x)=0$
	\end{observation}
\begin{definition}
	Una base de un $ A $-módulo $ M $ es un conjunto de generadores linealmente independiente.
	\[ M=<\{x_i\}_{i\in I}>=\sum_{i\in I} Ax_i, \{x_i\}_{i\in I} \text{ linealmente independientes.} \]
\end{definition}
\begin{definition}
	Un $ A $-módulo $ F $ es libre si tiene una base $\{x_i\}_{i\in I}$ que cumple:
	\begin{itemize}
		\item $x\in F\Rightarrow$ x se puede expresar como $x=\sum_{i\in I}a_i x_i$.
		\item Esta expresión es única, es decir, $\sum a_i x_i=\sum b_i x_i \Rightarrow \sum (a_i-b_i)x_i=0\Rightarrow (a_i-b_i)=0 \ \forall i$
	\end{itemize}
\end{definition}
\begin{observation}
	$ A $ cuerpo $ \Rightarrow $ todo $ A $-módulo es libre. \\
	Si $ A $ no es un cuerpo $ \Rightarrow $ existen módulos no libres.
\end{observation}
\begin{observation}
	Sea $ I $ un ideal propio no nulo de $ A $.\\
	$ M=A/I $ no es $ A $-módulo libre.
	Si $ \overline{x} \in A/I, 0\neq a\in I, 0\neq I \subset (0\overline{\lambda}), a\overline{x}=\overline{0} $
\end{observation}
\begin{example}
	\begin{itemize}
		\item $ A $ es $ A $-módulo libre base $ \{1\} $
		\item $ \{0\} $ es $ A $-módulo libre base $ \emptyset $
	\end{itemize}
\end{example}
JORGE 15-11

SERGIO 19-11

\begin{proposition}
	Sea $A=\mathbb{Z}$ y $M$ $A$-módulo con generadores $x_1,x_2$ y la relación $x_1+x_2=0$. Veamos que $Z\simeq M$.
\end{proposition}

\begin{proof}
	\ \\
	Vamos a definir un $\phi$ de forma que $\mathbb{Z}\xrightarrow{\ \phi\ }M$ sea isomorfismo.
	Definimos $\phi(a)=ax_1 \ \forall a \in \mathbb{Z}$.
	$\begin{rcases}
		x_1=\phi(1)\Rightarrow x_1\in Im(\phi) \\
		x_2=-x_1=-\phi(1)=\phi(-1)\Rightarrow x_2 \in Im(\phi)
	\end{rcases}$
	$\Rightarrow \phi$ es sobreyectiva. \\
	Veamos que $Ker(\phi)=\{0\}\Leftrightarrow \phi$ inyectiva.
\end{proof}
END SERGIO 19-11
BEGIN MANU 20-11
\end{document}
